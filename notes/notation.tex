\chapter*{Notation}
\addcontentsline{toc}{chapter}{Notation}

The symbol ``$\circledast$'' means that such definition or theorem was
created by the author, so it should be taken with care.

We usually refer to generic categories as $\mathcal C$ and similar uppercase letters with
curly font. For named categories (e.g. category of graphs, category of small categories,...)
we usually use a bold font with no italic, e.g. $\mathbf{Gr}, \mathbf{SmCat}, \mathbf{Top}$...

\begin{enumerate}
	\item $\circ$ - Used to symbolize composition of morphisms similar to how we compose functions.
	\item $\fatsemi$ - Represents composition, but with orders reversed, i.e. $g\circ f = f \fatsemi g$.
	\item $\diamond$ - \textit{Whiskering, prewhiskering, postwhiskering}.
	\item $\cong$ - Isomorphism in the context applied (e.g. categorical, set theoretical, etc).
	\item $\simeq$ - Equivalence of categories via natural transformations (weaker than isomorphism).
\end{enumerate}


