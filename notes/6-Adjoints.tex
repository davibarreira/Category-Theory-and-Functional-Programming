\chapter{Adjoints}

Adjoints, or, adjunctions, are not easily understood from the definition alone.
A good reference for a more intuitive description is the blog Math3ma, where some of the
intuition for adjoints was taken from. \citet{leinster2014basic} also gives a nice description.
There are several equivalent ways of defining adjunctions, and some are easier to grasp
than others. Before diving into rigorous definitions, we'll present some of the
intution behind.

\section{Intuitive Explanation for Adjoints}

Consider two functors $F:\mathcal C \to \mathcal D$ and $G:\mathcal D \to \mathcal C$.
We already saw that there is a weaker notion
of equivalence between categories(\ref{def:EquivalenceCategories}), where instead of
requiring the existence of an isomorphism given by
$F \circ G = id_\mathcal D$ and $F \circ G = id_\mathcal C$, we only
ask for $F \circ G \cong id_\mathcal D$ and $G \circ F \cong id_\mathcal C$.

We can go further and devise an even weaker notion of ``equality''
between categories.
This is where adjunctions come in. Instead
of isomorphisms, we can ask for the existence of two morphisms
(which in this context will be natural transformations),
where one morphism is $\varepsilon:F\circ G \to id_\mathcal D$ (called counit) and another morphism
is $\eta:id_\mathcal C \to G\circ F $ (called unit), and they are somehow related.

Another way to view adjunctions is the following. Given two functors $F:\mathcal C \to \mathcal D$
and $G:\mathcal D \to \mathcal C$. We say that $F$ is the left adjoint of $G$, which we write as $F \dashv G$,
if for every $C \in Ob_\mathcal C$ and every $D \in Ob_\mathcal D$, a morphism from $F(C)$ to $D$
is essentially the same as a morphism from $C$ to $G(D)$.

% Note that for $c \in Ob_\mathcal C$, then $\eta_c: c \to G(F(c))$ is $c \mapsto G(F(c))$. Similarly,
% $\varepsilon_{F(c)}: F(G(F(c))) \to F(c)$ is $F \circ G \circ F(c) \mapsto F(c)$.

% Now, there is a reason that our adjoints are either left or right. The adjunction
% is how we can ``translate'' between categories where one is not as ``expressive''
% as the other. How so? Consider for example the case of Linear Algebra, where
% for a given matrix $F \in \mathbb R^{n\times m}$ such that $n > m$. Now, suppose
% that there exists a matrix $G \in \mathbb R^{m \times n}$ such that
% $G$ is the pseudoinverse of $F$ and $G F = I_m$, i.e. the \textit{left inverse}.
% Since $G$ is the pseudoinverse, we have that $F G F = F$ and
% $G F G = G$.

% The pseudoinverse captures exactly the relation we were looking for. If $F$ and $G$
% are functors, we want to categorically specify that for them to be adjoints, it's
% required that $F \circ G \circ F = F$ and $G \circ F \circ G = G$.
% This can be done using the natural transformations. Note that we want to make
% \begin{align*}
%   F \circ G \circ F(c) &= F \circ \eta_c(c) = F(c) = \varepsilon_c(c), \\
%   G \circ F \circ G(F(c)) &= G\circ \varepsilon_{F(c)}(F(c)) = G(F(c)) = \eta_c(c),
% \end{align*}
% where we considered $\eta_c$ as a function, and $\eta_c(c) = G(F(c))$.
% Thus, the restriction we want to impose is that
% \begin{align*}
%   F \circ \eta_c(c) = \varepsilon_c(c), \quad
%   G\circ \varepsilon_{F(c)}(F(c)) = \eta_c(c).
% \end{align*}


% In this example, the category of vector spaces in $\mathbb R^m$ is ``smaller''
% than $\mathbb R^n$, yet, we want to somehow translate between them. The idea
% of how to properly translate between vector spaces is captured
% by inverse matrices. When it's not possible to find inverse, the \textit{best translation}
% is captured by the pseudoinverse, which is what the adjoint is trying to capture.
% The pseudoinverse $G$ is the left ajoint of $F$.

% Hopefully, this motivates the idea of adjoints. In a sense, one wishes to
% formalize this types of operations in which there is some sort of translation
% between categories with different expressiveness. We've shown how this is somehow
% related to weakening the requirement for the existence of isomorphisms.
% Now, let's indeed formalize a definition.

% But what this relation should be? Here an analogy from Linear Algebra might help.

% Remember the notion of a pseudoinverse of a matrix $A \in \mathbb R^{n\times m}$,
% denoted by $A^{\dagger} \in \mathbb R^{m \times n}$.
% We have that $A A^{\dagger}A = A$, and $A^{\dagger} A A^{\dagger} = A^{\dagger}$.
% How does this relate to our case? Note that $F\circ id_\mathcal C = F$, hence
% for $c \in Ob_\mathcal C$, we have $F(c) \in Ob_\mathcal D$, thus
% \begin{align*}
%   \varepsilon_{F(c)}:(F\circ G)(F(c))\to id_\mathcal D (F(c))\\
%   \varepsilon_{F(c)}:(F\circ G)(F(c))\to F(c).
% \end{align*}

% In the example of the pseudoinverse, the matrices are functors between
% vector spaces, 
% This means that somehow, via $\varepsilon$, we have $F G F = F$


\section{General Definitions}

The definition we are going to present assumes that both categories are locally small. This will
be enough for most applications. Yet, there are a more complicated definitions that also takes into
account categories that are not locally small. Remember that a category $\mathcal C$
is locally small if for every pair of objects $A$ and $B$, then $Mor_\mathcal C (A,B)$ is a set.

Before formalizing adjoints, let's present some helpful definitions. This is mostly based
on \citet{borceux1994handbook}.

\begin{definition}[Reflection]
	Let $F:\mathcal C \to \mathcal D$ be a functor and $d \in Ob_\mathcal D$.
	A reflection of $d$ along $F$ is a pair $(R_d, \eta_d)$ where
	\begin{enumerate}[(i)]
		\item $R_d \in Ob_\mathcal C$ and $\eta_d \in Mor_{\mathcal D}(d, F(R_d))$;
		\item if $c \in Ob_\mathcal C$ and $f \in Mor_{\mathcal D}(d,F(c))$, there exists
		      a unique morphism $g:R_d \to c$ such that
		      $F(g) \circ \eta_d = f$.
	\end{enumerate}
\end{definition}

\begin{proposition}
	Let $F:\mathcal C \to \mathcal D$ be a functor and $d \in Ob_\mathcal D$.
	When the reflection of $d$ along $F$ exists, it is unique up to an isomorphism.
\end{proposition}

\begin{proposition}
	Let $F:\mathcal C \to \mathcal D$ be a functor and assume that for every
	$d \in Ob_\mathcal D$, the reflection of $d$ along $F$ exists. Thus, there
	exists a functor $R:\mathcal D \to \mathcal C$ such that
	\begin{enumerate}[(i)]
		\item $\forall d \in Ob_\mathcal D, \ R(d) = R_d$;
		\item $(\eta_d:d \to FRd)_{d \in Ob_\mathcal D}$ is a natural transformation.
	\end{enumerate}
\end{proposition}

Now we can present the first definition of adjoint.
\begin{definition}[Adjoint]
	A functor $L:\mathcal D \to \mathcal C$ is a left adjoint to the functor
	$F:\mathcal C \to \mathcal D$ if there exists a natural transformation
	$\eta:id_{\mathcal D} \Rightarrow F \circ L$ such that for every
	$d \in \mathcal D$, the tuple $(Ld, \eta_d)$ is a reflection of $d$
	along $F$.
\end{definition}

\begin{definition}[Ajdoint]
	Let $\mathcal C$ and $\mathcal D$ be locally small categories. The adjoints (adjunction) between
	$\mathcal C$ and $\mathcal D$ are the functors
	\begin{displaymath}
		L:\mathcal C \to \mathcal D, \quad R:\mathcal D \to \mathcal C,
	\end{displaymath}
	together with a natural transformation $\alpha$ such that for every
	$A \in \mathcal C$ and $B, \in \mathcal D$,
	\begin{displaymath}
		\alpha_{A,B}: Mor_\mathcal D (L(A), B) \overset{\cong}{\longrightarrow} Mor_\mathcal C (A, R(B)),
	\end{displaymath}
	i.e. for every pair of objects $A \in \mathcal C$  and $B \in \mathcal D$, there is an isomorphism
	between the sets $Mor_\mathcal D(L(A), B)$ and $Mor_\mathcal C (A, R(B))$.
	Also,
\end{definition}

\begin{shaded}
	\textbf{Adjoints in Category Theory vs. Linear Algebra.} Names are suggestive in Mathematics,
	but even more so in Category Theory. Note that the definition above of an adjoint resembles
	somewhat the idea of adjoint linear transformation in Linear Algebra. Remember:
	\begin{displaymath}
		\langle A x, y \rangle  = \langle x, A^* y \rangle, \quad \forall x,y \in X,
	\end{displaymath}
	where $X$ is our vector space and $A: X \to X$ is a linear transformation.
\end{shaded}

\begin{proposition}
	Let $\mathcal C$ and $\mathcal D$ be two locally small categories, and $F:\mathcal C \to \mathcal D$
	a functor. If both functors $G,G':\mathcal D \to \mathcal C$ are right (or left) adjoints
	to $F$, then there exists a natural isomorphism between $G$ and $G'$.
\end{proposition}
