\chapter{Category Theory - The Very Basic}

The study of Category Theory enables us to view Mathematics from a vantage
point, and better understand how the different areas are connected.
By looking at the subject from the distance (via Category Theory), we get
a glimpse at the connections (and disconnections) between different fields.

Another interesting observation about Category Theory is that it's
breaking the theoretical barrier and it's starting to be applied
in real world applications. One prominent example is in programming,
as highlighted by the book \citet{milewski2018category}.
The book by \citet{fong2019invitation} also focus on the application
side of Category Theory.

This chapter introduces the field of Category Theory by
presenting the first three components of the theory, i.e.
categories, objects, and morphisms.

\section{Universes, Sets and Classes}

This section is a bit technical. The goal is to formally
define the difference between a set and a class. This is a nuanced
point, but a necessary one if we wish to do things with rigour.

When working on Category Theory, it's common to find universal
statements such as ``for all topological spaces...''. The issue
with such statements is that, in a purely set-theoretical sense,
we have to know whether such large collection (``all topological spaces'')
is indeed a set. We might be tempted to say that's true, but
it's not so simple. Russel's Paradox of whether there is a set of all sets
is an example of when the answer is ``no''.

Therefore, in order to deal with such issue, we need a way to differentiate
between a valid set and an arbitrary collection. Here is where the notion of a Universe comes in.

\begin{definition}[Universe]
	We say that a set $\mathfrak U$ is a universe if\footnote{Definition from \citet{borceux1994handbook}}
	\begin{enumerate}[(i)]
		\item $x\in y$ and $ y\in \mathfrak U$, then $x \in \mathfrak U$;
		\item $I \in \mathfrak U$, and $\forall i \in I, x_i \in \mathfrak U$, then $\cup_{i\in I}x_i \in \mathfrak U$;
		\item $x \in \mathfrak U$ then $\mathcal P(x) \in \mathfrak U$, where $\mathcal P(x)$ is the power set;
		\item $x \in \mathfrak U$ and $f:x\to y$ is a surjective function, then $y \in \mathfrak U$;
		\item $\mathbb N \in \mathfrak U$.
	\end{enumerate}
\end{definition}

From this definition, one can prove the following proposition.
\begin{proposition}
	Let $\mathfrak U$ be a universe. Therefore:
	\begin{enumerate}[(i)]
		\item $x \in \mathfrak U$ and $y \subset x$, then $y \in \mathfrak U$;
		\item $x \in \mathfrak U$ and $y \subset x$, then $\{x,y\} \in \mathfrak U$;
		\item $x \in \mathfrak U$ and $y \subset x$, then $x\times y \in \mathfrak U$;
		\item $x \in \mathfrak U$ and $y \subset x$, then $y^x \in \mathfrak U$, where $y^x$ is the set of functions $f:x \to y$.
	\end{enumerate}
	\label{prop:universe}
\end{proposition}

With this definition, we state the axiom of existence of universes.

\begin{axiom}
	Every set $S$ belongs to some universe $\mathfrak U$.
\end{axiom}

\begin{definition}
	For a fixed universe $\mathfrak U$, if a set $S$ is an element of $\mathfrak U$,
	then $S$ is called a \textit{small set}.
\end{definition}

Talking about ``small sets'' and ``big set'' might become daunting, so instead, we
use a different convention which is based on Gödel-Bernays theory of sets and classes.
This theory states that:
\begin{axiom}[Gödel-Bernays]
	A class is a set if and only if it belongs to some (other)
	class.
	\label{axiom:gb}
\end{axiom}

Note that using the notion of Universes, we can recover Gödel-Bernays theory. For that,
use the following definition:
\begin{definition}
	For a fixed universe $\mathfrak U$, we call $S$ a \textit{set} if it's an element
	of $\mathfrak U$, and call $S$ a \textit{class} if it's a subset of $\mathfrak U$.
	A class that is not a set is called a \textit{proper class}.
\end{definition}

From now on, whenever we say \textit{set} we are implying \textit{small set}
and whenever we say \textit{class} we are implying either small or big sets,
following \citet{borceux1994handbook} convention.


\section{What is a Category?}

Let's formally define a category and provide some examples.

\begin{definition}[Category]
	A category $\mathcal C = \langle Ob_{\mathcal C}, Mor_{\mathcal C} \rangle$ consists
	of a class of objects $Ob_\mathcal C$ and a class of morphisms
	$Mor_\mathcal C$ satisfying the following conditions:
	\begin{enumerate}[(i)]
		\item Every morphism $f \in Mor_\mathcal C$ is associated to two objects $X,Y \in Ob_{\mathcal C}$
		      which is represented by $f:X \to Y$ or $X \xrightarrow{\hspace{3mm}f \hspace{3mm}} Y$,
		      where $dom(f) = X$ is called the domain of $f$ and $cod(f)=Y$ is the codomain. Moreover, we define
		      $Mor_\mathcal C (X,Y)$ as
		      \begin{displaymath}
			      Mor_\mathcal C (X,Y) := \{f \in Mor_\mathcal C \ : \ X \in dom(f), \ Y \in cod(f)\};
		      \end{displaymath}
		\item For any three objects $X,Y, Z \in Ob_\mathcal C$, there exists a composition operator
		      \begin{displaymath}
			      \circ: Mor_\mathcal C (X,Y)   \times Mor_\mathcal C (Y,Z) \to Mor_\mathcal C (X,Z),
		      \end{displaymath}
		\item For each object $X \in Ob_\mathcal C$ there exists a morfism $id_X \in Mor_\mathcal C (A,A)$
		      called the identity.
	\end{enumerate}
	The composition operator must have the following properties:
	\begin{enumerate}[(p.1)]
		\item \textit{Associative}: for every $f \in Mor_\mathcal C (A,B),
			      g \in Mor_\mathcal C (B,C), h \in Mor_\mathcal C (C,D)$ then
		      \begin{displaymath}
			      h \circ (g \circ f) = (h \circ g) \circ f.
		      \end{displaymath}
		\item For any $f \in Mor_\mathcal C (X,Y)$, $g \in Mor_\mathcal C (Y,X)$,
		      \begin{displaymath}
			      f \circ id_X = f,  \quad id_Y \circ g = g.
		      \end{displaymath}
	\end{enumerate}
\end{definition}

There are many ways to refers to the class of morphisms $Mor_\mathcal C (X,Y)$, such as
$\mathcal C(X,Y)$ or $\text{hom}_\mathcal C (X,Y)$. The reason for this is that
this set is sometimes called hom-set. In these notes, we'll use either $Mor_\mathcal C (X,Y)$
or $\mathcal C (X,Y)$ when there is no ambiguity. Also, we'll use $dom_f$ to mean $dom(f)$,
and similarly for the codomain.

Another point about conventions. When talking about composition, it's convenient
to use the operator $\fatsemi$, which is equivalent to the composition $\circ$,
but with the inverted order, i.e. $f \fatsemi g = g \circ f$. The convinience
will become clearer once we introduce Hasse diagrams as a way to represent categories.

\begin{definition}[Small Category]
	A category $\mathcal C$ is \textit{small} if $Ob_\mathcal C$ and
	$Mor_\mathcal C$ are sets (small).
\end{definition}

\begin{definition}[Locally Small Category]
	A category $\mathcal C$ is \textit{locally small} if
	for any $A, B \in Ob_\mathcal C$, then
	$Mor_\mathcal C(A,B)$ is a set (small).
\end{definition}

\subsection{Examples of Categories}

It's very common to represent categories via Hasse Diagrams. In these diagrams, the
objects are represented as dots, and the morphisms as arrows. Let's show some examples.

\begin{example}[Category $\bm 1$]
	The category $\bm 1$ consists of $Ob_{\bm 1} := \{A\}$ and $Mor_{\bm 1} = \{id_A\}$.
	The diagram for such category is shown below.
	\begin{figure}[H]
		\begin{center}
			\includegraphics{./notebooks/1Cat}
		\end{center}
		\caption{Hasse diagram of category $\bm 1$.}
		\label{fig:1Cat}
	\end{figure}
\end{example}

\begin{example}[Category $\bm 2$ and $\bm 3$]
	The category $\bm 2$ consists of $Ob_{\bm 2} := \{A, B\}$ and $Mor_{\bm 1} = \{id_A, id_B, f\}$,
	where $f:A \to B$.
	The diagram for such category is shown below.
	\begin{figure}[H]
		\begin{center}
			\includegraphics{./notebooks/2Cat}
		\end{center}
		\caption{Hasse diagram of category $\bm 2$.}
		\label{fig:2Cat}
	\end{figure}

	Since we know that identities are always present in categories, we'll
	omit them from future diagrams when there is no ambiguity. Thus,
	the figure below represents the same diagram as Figure \ref{fig:2Cat}.
	\begin{figure}[H]
		\begin{center}
			\includegraphics{./notebooks/2Catsimple}
		\end{center}
		\caption{Hasse diagram of category $\bm 2$ omitting identity morphisms.}
		\label{fig:2Catsimple}
	\end{figure}

	The category $\bm 3$ has three morphisms besides the identities, given
	by $f$, $g$ and their composition $f \fatsemi g$. The figure below
	illustrates the category with all it's morphisms.

	\begin{figure}[H]
		\begin{center}
			\includegraphics{./notebooks/3CatComplete}
		\end{center}
		\caption{Hasse diagram of category $\bm 3$ showing all morphisms.}
		\label{fig:3Catcomplete}
	\end{figure}

	Drawing all the morphisms can make the diagram become too crowded, specially
	as the number of objects and morphisms grows. Hence, we simplify the
	diagram representation by ommiting not only the identity morphisms, but also
	the compositions. These can always be assumed to exist, since they are a necessery
	condition for every category.
	Thus, the figure below represents the same diagram as Figure \ref{fig:3Catcomplete}.

	\begin{figure}[H]
		\begin{center}
			\includegraphics{./notebooks/3Catsimple}
		\end{center}
		\caption{Hasse diagram of category $\bm 3$ omitting identities and compositions.}
		\label{fig:3Catsimple}
	\end{figure}

\end{example}

\begin{example}[Discrete Categories]
	The discrete category $\mathbf{\underline{N}}$ is the category with $N$ objects
	and $Mor_{\mathbf{\underline{N}}} := \{id_1,...,id_N\}$. An example of this category is
	illustrated below.

	\begin{figure}[H]
		\begin{center}
			\includegraphics{./notebooks/3Discrete.pdf}
		\end{center}
		\caption{Example of category $\mathbf{\underline{3}}$.}
		\label{fig:3Discrete}
	\end{figure}
\end{example}

\begin{example}[Preorders]
	A preorder is defined by a tuple $(P, \leq)$, where $P$ is a set of values, such that
	\begin{enumerate}[(i)]
		\item For $a,b \in P$, if $a\leq b$ and $b \leq c$, then $a \leq c$;
		\item For every $a \in P$, $a \leq a$.
	\end{enumerate}
	We can show that actually, this is a category, which we'll call $\mathcal P$,
	where $Ob_\mathcal P = P$ and each morphism $f$ represents $a \leq b$, where
	$cod_f = a$ and $dom_f = b$.
	One example of preorder is the set of $\mathbb N$ equiped with the binary relation $\leq$
	which is shown in the diagram below.

	\begin{figure}[H]
		\begin{center}
			\includegraphics{./notebooks/NCat}
		\end{center}
		\caption{Hasse diagram of preorder category of natural numbers.}
		\label{fig:NCat}
	\end{figure}
\end{example}

Note that in preorders, there is at most one morphism between each pair of objects.
Thus, categories with such property are often referred as \textit{thin categories}
or \textit{preorder category} (in \citet{fong2019invitation}, the authors call this
a \textit{preorder reflection}).

\begin{example}[Monoids]
	A monoid is a triple $(M, \oplus, e_M)$ where $\oplus:M\times M \to M$ is a binary operation
	and $e_M$ the neutral element, such that:
	\begin{enumerate}
		\item $a \oplus (b \oplus c) = (a \oplus b) \oplus c$
		\item $a \oplus e_M = e_M \oplus a = a$.
	\end{enumerate}
	Note that $(\mathbb N \cup\{ 0\}, +, 0)$ is a monoid.

	Moreover, each single monoid can be defined as a category itself. For
	monoid $(M, \oplus, e_M)$, define a category $\mathcal C$
	such that $Ob_\mathcal C := \{M\}$, and the set of morphisms
	are the elements of $M$, i.e.
	$Mor_\mathcal C := \{a \in M\}$. Finally, we define the composition
	operartion as $a \circ b := a \oplus b$
	Thus, $(\mathbb N \cup \{0\}, +, 0)$ is the category illustrated in
	Figure \ref{fig:NMonoid}.

	\begin{figure}[H]
		\begin{center}
			\includegraphics{./notebooks/NMonoid}
		\end{center}
		\caption{Hasse diagram of monoid category of natural numbers.}
		\label{fig:NMonoid}
	\end{figure}

\end{example}

There are many other examples:
\begin{enumerate}[1.]
	\item $\mathbf{Set}$ which is the category of sets, where the objects are sets and the morphisms are functions between sets.
	\item $\mathbf{Top}$ is the category where topological spaces are the objects and continuous functions are the morphisms.
	\item $\mathbf{Vec}_\mathbb F$ is the category where vector spaces over field $\mathbb F$ are the objects,
	      and linear transformations are the morphisms.
	\item $\mathbf{Gr}$ is the category of directed graphs, where $Ob_{\mathbf{Gr}} := \{\text{Vertex}, \text{Arrow}\}$,
	      and the morphisms are
	      \begin{displaymath}
		      Mor_{\mathbf{Gr}} := \{
		      src,
		      tgt,
		      id_{\text{Vertex}},
		      id_{\text{Arrow}}
		      \}
	      \end{displaymath}
	      where $src:\text{Arrow} \to \text{Vertex}$ returns the source vertex for each arrow and
	      $tgt:\text{Arrow} \to \text{Vertex}$ returns the target vertex.
\end{enumerate}


\subsection{Isomorphism, monomorphism and epimorphism}

A very important definition in Category Theory is the notion of isomorphism.
In Set Theory, we say that two sets are isomorphic if there is a bijective
function between them. Yet, this concept is not restricted to Set Theory
and can be generalized in Category Theory as follows:

\begin{definition}[Categorical Isomorphism]
	Let $\mathcal C$ be a category with $X,Y \in Ob_\mathcal C$ and $f \in Mor_\mathcal C (X,Y)$.
	\begin{enumerate}[(i)]
		\item We say that $f$ is \textit{left invertible} if there exists $f_l \in Mor_\mathcal C (Y,X)$ such
		      that $f_l \circ f = id_X$;
		\item We say that $f$ is \textit{right invertible} if there exists $f_r \in Mor_\mathcal C (Y,X)$ such
		      that $f \circ f_r = id_Y$;
		\item We say that $f$ is invertible if it's both left and right invertible.
	\end{enumerate}
	When an invertible morphism exists between $X$ and $Y$, we say that they are isomorphic.
	\label{def:CategoricalIsomorphism}
\end{definition}

\begin{proposition}
	The following properties on inverses are true:
	\begin{enumerate}[1.]
		\item If $f$ is an invertible morphism, then the left and right inverses are the same.
		\item If $f$ and $g$ are invertible and composable, then $f \circ g$ is also invertible.
	\end{enumerate}
\end{proposition}
\begin{proof}
	1. Let $f$ be invertible with left inverse $f_l$ and right inverse $f_r$. Therefore,
	\begin{displaymath}
		f_l \circ id_Y = f_l \circ f \circ f_r = id_X \circ f_r = f_r.
	\end{displaymath}

	2. Let $f:A\to B$ and $g: B \to C$ be invertible and composable, with $f\circ g: A \to C$.
	There exists the inverses $g^{-1}: C \to B$ and $f^{-1}:B \to A$. Note that
	$f^{-1}\circ g^{-1}:C \to A$, thus
	\begin{align*}
		(f^{-1} \circ g^{-1}) \circ (g \circ f) =
		f^{-1} \circ (g^{-1} \circ g) \circ f =
		(f^{-1} \circ id_B) \circ f =
		f^{-1} \circ f =
		id_A.
	\end{align*}
	\begin{align*}
		(g \circ f) \circ (f^{-1} \circ g^{-1})  =
		g \circ (f \circ f^{-1}) \circ g^{-1} =
		g \circ (id_B \circ g^{-1}) =
		g \circ g^{-1} =
		id_B.
	\end{align*}
	We conclude that $(g\circ f)^{-1} = f^{-1} \circ g^{-1}$.

\end{proof}
Although similar to set isomorphism, categorical isomorphism is in a sense more general,
and captures our intuition of isomorphism between categories better than the set theoretic case,
even when we have finite objects.

Consider the following example.
Let $\mathcal P$ be the category of Posets, where posets $(P,\leq_p)$ are the objects.
Take two objects $P_1, P_2 \in Ob_\mathcal P$,
where $P_1:=\{a,b\}$ with $a$ and $b$ \textbf{not} comparable,
and $P_2:=\{0,1\}$ where indeed $0 \leq 1$. The question is whether $P_1$ is ``actually'' isomorphic
to $P_2$, and our intuition say that they should not be, since
$P_1$ has two incomparable elements while $P_2$ has two comparable elements.

If we use the set theoretic definition, we would conclude that they \textbf{are} isomorphic,
since there is a bijective function between $P_1$ and $P_2$.
Take for example $f:P_1\to P_2$ where $f(a)=0$ and $f(b)=1$.
So the set theoretic isomorphism does not capture what we want. What about the categorical isomorphism?
We can prove that this will not be an isomorphism using the categorical definition. Yet,
in order to prove this, we need to specify what are the morphisms between the posets,
and to do this, we need to define what are functors.

In the same way that set isomorphism is not the same as categorical isomorphism,
the notions of injectivity and surjectivity are not equivalent to their categorical
counterparts, which are called monomorphism and epimorphism.

\begin{definition}[Monomorphism]
	Let $\mathcal C$ be a category and $f \in Mor_\mathcal C(A,B)$. We say that
	$f$ is a monomorphism (or monic), if
	\begin{displaymath}
		f \circ g = f \circ h \implies g = h.
	\end{displaymath}
\end{definition}

\begin{definition}[Epimorphism]
	Let $\mathcal C$ be a category and $f \in Mor_\mathcal C(A,B)$. We say that
	$f$ is an epimorphism (or epic), if
	\begin{displaymath}
		g \circ f = h \circ f \implies g = h.
	\end{displaymath}
\end{definition}
\textbf{Important!} A morphism $f$ can be both epic and monic, without being
an isomorphism, which again highlights the difference between this concepts and their
set-theoretic counterparts. A counter example for this is the inclusion
$\mathbb Z \hookrightarrow \mathbb Q$ in the category $\mathbf{Ring}$.
This category has rings as objects and ring homomorphisms as morphisms.
Note that the inclusion is a homomorphism from $\mathbb Z$ to $\mathbb Q$ that is both monic and epic,
yet, there is no homomorphism from $\mathbb Q$ to $\mathbb Z$.

\begin{proposition}
	The following properties on monomorphism and epimorphism are true:
	\begin{enumerate}[1.]
		\item $f$ left-invertible $\implies$ $f$ is monic. The converse is not true.
		\item $f$ right-invertible $\implies$ $f$ is epic. The converse is not true.
		\item $f$ invertible $\implies$ $f$ is monic and epic. The converse is not true.
		\item $f$ monic and right-invertible $\implies $ $f$ is isomorphism.
		\item $f$ epic and left-invertible $\implies $ $f$ is isomorphism.
	\end{enumerate}
\end{proposition}

\begin{proof}
	1. Note $f:A \to B$ left-invertible implies that there exists a $f_l:B \to A$ such that
	$f_l \circ f = id$. Hence, for a $g:B\to C$ and $h: B \to C$, if
	\begin{displaymath}
		f \circ g = f \circ h,
	\end{displaymath}
	then we have that
	\begin{displaymath}
		f_l \circ f \circ g = f_l \circ f \circ h \implies g =h.
	\end{displaymath}
	To show that the converse is false, consider the category $\mathbf{2}$ (Figure \ref{fig:2Cat}). Note that
	$f:A\to B$ is monic, since the only morphism that composes with $f$ is
	$id_A$ and $id_B$. Yet, note that $f$ is not left invertible, since there isn't even
	a morphism from $B$ to $A$.

	2. Use the same argument, but reversing the order of the compositions.
	For the converse, again consider the same category $\mathbf{2}$. Note that
	$f:A\to B$ is epic, but it's not right invertible.

	3. True since invertible means left and right invertible.

	4. Since $f:A \to B$ right invertible, then there exists $f_r:B \to A$
	such that $f \circ f_r = id_B$. Thus,
	\begin{displaymath}
		id_B \circ f = (f \circ f_r) \circ f =
		f \circ (f_r \circ f) =
		f \circ id_A \implies f_r \circ f = id_A.
	\end{displaymath}
	5. Same argument.
\end{proof}

Lastly, let's introduce two definitions that are helpful when talking about morphisms.
\begin{definition}[Endomorphism and Automorphism]
	A morphism is called an endomorphism is the domain and codomain object are the same.
	If this endomorphism is an isomorphism, then we call it an automorphism.
\end{definition}

\subsection{Zero, Initial and Terminal Objects}

\begin{definition}[Zero, Initial and Terminal]
	Let $\mathcal C$ be a category.
	\begin{enumerate}[1.]
		\item An object $I \in Ob_\mathcal C$ is \textit{initial} if for every $A \in Ob_\mathcal C$,
		      there is exactly one morphism from $I$ to $A$. Thus, from $I$ to $I$ there is only the identity.
		\item An object $T \in Ob_\mathcal C$ is \textit{terminal} if for every $A \in Ob_\mathcal C$,
		      there is exactly one morphism from $A$ to $T$. Thus, from $I$ to $I$ there is only the identity.
		\item An object is \textit{zero} if it's both terminal and initial.
	\end{enumerate}
\end{definition}

Note that in the definitions above, we are defining these objects in terms of existence and
uniqueness of morphisms, which is known in category theory as \textbf{universal constructions}
(more on this later).

\begin{theorem}
	Every \textit{initial} object is unique up to an isomorphism, i.e. if in a category there
	are two \textit{initial} objects, then they are isomorphic.
	Similarly, \textit{terminal} objects are unique up to an isomorphism.
	Moreover, the isomorphism is unique between initial object, and between terminal objects.
\end{theorem}
\begin{proof}
	Let $I_1, I_2$ be initial. Then, there exists only $f:I_1 \to I_2$ and $g:I_2 \to I_1$.
	But since $g \circ f:I_1 \to I_1$ is a morphism from the initial object $I_1$, it must
	be equal to $id_{I_1}$. The same for $I_2$, which implies that $f$ and $g$ are inverses,
	and thus the objects are isomorphic. Since both $f$ and $g$ are the only morphisms from
	$I_1$ and $I_2$, this also implies that their are the only isomorphism.
	The same proof works for terminal objects.
\end{proof}


\begin{example}[Terminal and Initial Objects in $\mathbf{Set}$]
	Without thinking too much, one might assumet that in the category $\mathbf{Set}$
	the empty set would be a zero object; but that's not true.
	In reality, the empty set is the initial object, since $f:\varnothing \to B$
	is the only function from the empty set to any other set. Why is this valid?

	Remember that in set theory, a function from two sets is defined as a binary
	relation such that for every $x \in dom_f$, there is a unique $y \in cod_f$, i.e.
	$f$ is a triple $(A,B,G)$, where $A = dom_f$, $B = cod_f$ and $G \subset A \times B$
	such that $\forall x \in dom_f$, there exists a unique $y \in B$, such that $(x,y) \in G$.

	Since $dom_f = \varnothing$, we have that $G \subset \varnothing \times B$, but this
	is actually empty. Why? If $\varnothing \times B$ is not empty, then there exists
	$(a,b) \in \varnothing \times B$, which is false, since this would imply that
	$a \in \varnothing$, which contradicts the definition of the empty set that says
	that it can have no elements (note that $\varnothing \in \varnothing$ is actually false).

	With this, we have that $G = \varnothing$, thus, the only possible function from
	$\varnothing$ to $B$ is $f = (\varnothing, \varnothing, B)$. Which proves that the
	empty set is initial.

	But what about terminal? The empty set actually does not have any morphisms
	that arrives on it, since there is no function $f:A \to \varnothing$.
	The terminal sets in $\mathbf{Set}$ are actually all the singletons (sets with only
	one element), since for any $\{a\}$, there will be only one function
	$g: A \to \{a\}$, which is $g(x) = a$.
	\label{ex:InitialTerminalSet}
\end{example}

Another definition we have is that of a \textit{zero} morphism. The idea here
is that this morphism must take the elements of an object $A$ to the
zero element in $B$, for example, a for two vector spaces $\mathbb R^n$ and $\mathbb R^m$,
the zero linear transformation $z:\mathbb R^n \to \mathbb R^m$ should take every vector
$n$-dimensional vector to the $\mathbf{0}$ $m$-dimensional vector. In Category
Theory we do not talk about morphisms according to how they act on the elements, but
only in the objects. So we cannot define $z$ by saying to which element it maps.
Yet, there is a way to do this in Category Theory, which gives rise to the \textit{zero} morphism
definition.
\begin{definition}[Zero Morphism]
	Let $\mathcal C$ be a category, and $0$ be a zero object.
	A morphism $z:A \to B$ is a zero morphism if there exists two morphisms
	$f:A\to 0$ and $g:0 \to B$, such that
	\begin{displaymath}
		z = g \circ f.
	\end{displaymath}
\end{definition}
See that this makes intuitive sense. In our example, since we wish to take
$v \in \mathbb R^n$ to $0 \in \mathbb R^m$, we first take all $v$ to the zero object,
which in the category of vector spaces will be the zero vector space, i.e. $\{0\}$ the space
where $0 \in \mathbb R$ is the only element. So now all $v$ are $0$. Note that
every linear transformation from $\{0\}$ to $\mathbb R^m$ must take $0$ to $\mathbf{0} \in \mathbb R^m$,
otherwise, suppose that $g(0)=\mathbf{v} \neq \mathbf{0}$, hence for a scalar $\alpha$,
\begin{displaymath}
	g(0) = g(\alpha0) = \mathbf{v} \neq \alpha \mathbf{v} = \alpha g(0).
\end{displaymath}
This is a contradiction, since $g$ is a linear transformation.

\begin{theorem}
	Let $\mathcal C$ be a catgory with zero object $0$.
	Then there exists a unique zero morphism between any two objects.
\end{theorem}
\begin{proof}
	Let $A, B \in Ob_\mathcal C$.
	By the definition of the zero object, there exists a unique
	$f:A \to 0$ and $g:0\to B$, thus, $g \circ f$ is a zero morphism
	by definition and is unique, since there is no other $f$ and $g$ with these respective
	domain and codamain.

	Moreover, note that if $z:A \to B$ is our zero morphism and $h:B \to C$, then
	\begin{displaymath}
		h \circ z = h \circ (g \circ f) = (h \circ g) \circ f.
	\end{displaymath}
	But, $l =(h \circ g):0 \to C$, which means that $l \circ f$ is a zero morphism.
	The same argument works with a composition from the other direction. This means
	that compositions with zero morphisms return zero morphisms.
\end{proof}

\subsection{Understanding Duality}

In several fields of Mathematics, one is faced with the informal notion of a dual.
Mathematicians define a concept, and call them the dual in some sense, for example,
the dual vector space, the dual of an optimization problem, and many more.
I always found puzzling what exactly held these things together, i.e.
what was the underlying principle that made something a dual of another.

Fortunately, Category Theory has a very elegant answer. For a given
category $\mathcal C$, the dual category is denoted by
$\mathcal C^{op}$ where are the objects are the same, but the morphisms
are inverted. This means that $Ob_{\mathcal C} = Ob_{\mathcal C^{op}}$,
and for every $f\in Mor_\mathcal (A,B)$, we have $f^{op} \in Mor_{\mathcal C^{op}}(B,A)$.

This definition gives rise to a very interesting result (observation), called the
\textit{Duality Principle}.

\begin{definition}[Dual Property and Dual Statement]
	We say $p^{op}$ is the dual property for $p$ if for all categories
	\begin{displaymath}
		\mathcal C \text{ has }p^{op} \iff
		\mathcal C^{op} \text{ has }p.
	\end{displaymath}

	For a statement $s$ about a category $\mathcal C$, the dual statement is the same
	statement, but with regards to $\mathcal C^{op}$.
	\label{def:DualProperty}
\end{definition}
For example, ``a category has an initial object if and only if the dual category has a terminal object''.
In this example, the property of having an initial object is the dual property of having a terminal object,
since the above statement is true for any category.
What about the dual statement? The dual for the statement ``the category $\mathcal C$ has an initial object''
is ``the category $\mathcal C^{op}$ has an initial object''. Note that the dual statement is not always true.
And here is where we get the duality principle.

\begin{definition}[Duality Principle]
	If a statement $s$ is true for every category, then the dual statement is also true for every
	category.
\end{definition}
Let's digest a bit what this principle states. If we can prove that a certain statement is
true for any arbitrary category, then it's dual will also be true without any effort what so ever.
\citet{roman2017introduction} gives a nice example of this. We already prove that for any
category, if an initial object exists, this initial object is unique up to an isomorphism.
Note that this is a statement that is true for any category, so the duality principle applies,
i.e. the dual statement is true. And what is the dual statement? That for every $\mathcal C^{op}$
the initial object is unique up to an isomorphism. But an initial object in $\mathcal C^{op}$
is a terminal object in $\mathcal C$. So we have, without any effort, that every terminal
object is unique up to an isomorphism.

\subsection{Categorical Product and Coproduct}

In set theory, we are used to the notion of a Cartesian product. Similarly
to how we did for isomorphism, the idea of a product can be generalized via
Category Theory. Here is how it's done.

\begin{definition}[Span]
	Let $A,B$ be objects in a  category $\mathcal C$. A span
	on $A$ and $B$ is a triple $(Z,f,g)$ where $f:Z\to A$ and
	$g:Z\to B$ are morphisms in $\mathcal C$. This is
	shown in figure \ref{fig:Span}.
	\label{def:Span}
\end{definition}

\begin{figure}[H]
	\begin{center}
		\includegraphics{./notebooks/Span.pdf}
	\end{center}
	\caption{Diagrams showcasing a span between $A$ and $B$.}
	\label{fig:Span}
\end{figure}

\begin{definition}[Categorical Product]
	Let $A,B$ be objects in a  category $\mathcal C$. A span $(A\times B, \pi_1, \pi_2)$
	is called a product between $A$ and $B$ if for every span $(Z, f, g)$ of $A$ and $B$,
	there exists a unique morphism $h_{f,g}:Z \to A \times B$ such that
	$h_{f,g} \fatsemi \pi_1 = f$ and $h_{f,g} \fatsemi \pi_2 = g$. That is the same
	as saying that the diagram \ref{fig:Product} commutes.
\end{definition}

\begin{figure}[H]
	\begin{center}
		\includegraphics{./notebooks/CategoricalProduct.pdf}
	\end{center}
	\caption{Diagrams showcasing the categorical product. Note that the dashed line
		is intended to highlight that the morphism $h_{f,g}$ is uniquely induced by $f$ and $g$.}
	\label{fig:Product}
\end{figure}

Note that this definition of a product is a \textbf{universal construction}, since
it's done via existence and uniqueness.
Another important aspect to note is that not every pair of objects in a category might have a
product associated.

\begin{theorem}
	For a category $\mathcal C$, a pair of objects $A$ and $B$ can have more than one
	product construction, but if this is the case, then both these constructions will be
	isomorphic.
\end{theorem}

\begin{proposition}[Categorical Product vs Set Product]
	The categorical product generalizes the Cartesian product in set theory.
\end{proposition}
\begin{proof}
	Consider the span $(X\times Y, \pi_1,\pi_2)$ where $\pi_1(x,y) = x$ and $\pi_2(x,y) = y$.
	Thus, for any span $(Z, f, g)$ of $A$ and $B$, make $h_{f,g}(z) = (f(z), g(z)) \in X \times Y$.
	This is how the Cartesian product works.

	Let's drop the subscript in $h_{f,g}$.
	Now we have to show that $h$ is a unique morphism,
	and that $h \fatsemi \pi_1 = f$ and $h \fatsemi \pi_2 = g$.

	The second condition is trivially true, just note that
	\begin{displaymath}
		\forall z \in Z, \ \pi_1(h(z)) = \pi_1((f(z), g(z))) = f(z) \implies
		h \fatsemi \pi_1 = f,
	\end{displaymath}
	and the same argument works for $\pi_2$ and $g$.

	For uniqueness, consider $h':Z\to X\times Y$ such that
	$h' \fatsemi \pi_1 = f$, and $h' \fatsemi \pi_2 = g$.
	If $h' \neq h$, then there is a $z \in Z$,
	such that $h(z) = (f(z), g(z)) \neq h'(z)$. Bu then,
	$\pi_1(h'(z)) \neq f(z)$ or $\pi_2(h'(z)) \neq g(z)$, which
	is a contradiction.

\end{proof}

From the definition of a product, it's easy to think of possible dual constructions
by just inverting the arrows (morphisms).

\begin{definition}[Cospan]
	Let $A,B$ be objects in a  category $\mathcal C$. A cospan
	on $A$ and $B$ is a triple $(Z,f,g)$ where $f:A\to Z$ and
	$g:B\to Z$ are morphisms in $\mathcal C$. This is
	shown in Figure \ref{fig:Cospan}.
\end{definition}

\begin{figure}[H]
	\begin{center}
		\includegraphics{./notebooks/Cospan.pdf}
	\end{center}
	\caption{Diagrams showcasing a cospan between $A$ and $B$.}
	\label{fig:Cospan}
\end{figure}

\begin{definition}[Categorical Coproduct]
	Let $A,B$ be objects in a  category $\mathcal C$. A cospan $(A + B, i_1, i_2)$
	is called a product between $A$ and $B$ if for every span $(Z, f, g)$ of $A$ and $B$,
	there exists a unique morphism $h_{f,g}:Z \to A \times B$ such that
	$i_1 \fatsemi h_{f,g} = f$ and $i_2 \fatsemi h_{f,g} = g$. That is the same
	as saying that the diagram \ref{fig:Coproduct} commutes.
\end{definition}

\begin{figure}[H]
	\begin{center}
		\includegraphics{./notebooks/CategoricalCoproduct.pdf}
	\end{center}
	\caption{Diagrams showcasing the categorical coproduct. Note that the dashed line
		is intended to highlight that the morphism $h_{f,g}$ is uniquely induced by $f$ and $g$.}
	\label{fig:Coproduct}
\end{figure}

While the categorical product was constructed to generalize the Cartesian set product,
the coproduct was constructed in Category Theory, so the question is
``to what does the coproduct corresponds in set theory?''. The answer is the
disjoint union! It's not a coincidence that we used ``$+$'' to symbolize it.

The idea of a product construction induced the notion of a product object
$A \times B$. Yet, this construction also induces another definition, of the
so called (co)product morphism.

\begin{definition}[Product and Coproduct Morphism]
	Let $A, B, C, D \in Ob_\mathcal C$,
	$(A\times B, \pi^{A,B}_1, \pi^{A,B}_2)$ and
	$(C\times D, \pi^{C,D}_1, \pi^{C,D}_2)$ two products in the category $\mathcal C$,
	$f:A\to C$ and $g:B \to D$ two morphisms in $\mathcal C$. Hence, the
	morphism induced by $\pi_1^{A,B} \fatsemi f$ and $\pi_2^{A,B} \fatsemi g$ is
	called product morphism and denoted by $f \times g$, where
	\begin{displaymath}
		f \times g := (\pi_1^{A,B} \fatsemi f, \pi_2^{A,B} \fatsemi g): A\times B \to C \times D.
	\end{displaymath}
\end{definition}

\begin{theorem}
	The product morphism $f\times g$ is the only morphism in $Mor_\mathcal C (A\times B, C \times D)$
	such that
	\begin{displaymath}
		\pi_1^{C,D} \circ (f\times g) = f \circ \pi_1^{A,B}, \quad
		\pi_2^{C,D} \circ (f\times g) = g \circ \pi_2^{A,B}.
	\end{displaymath}
\end{theorem}

The coproduct morphism follows the same definition, but with coproducts.

Finally, one might be wondering what is an actual example of a product morphisms.
For sets, it's the intuitive object, e.g. for two functions $f:A \to C$
and $g:B \to D$, the product morphism $f\times g:A\times B \to C \times D$
is just $(f\times g) (x,y) = (f(x), g(y))$.

\subsection{Cartesian Categories}

\subsection{Pullback, Pushout and Equalizers}
