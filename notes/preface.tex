\newpage
\chapter*{Preface}

The goal of these notes is to introduce Category Theory with mathematical rigour, and then
show how this theory is used within the Functional Programming paradigm.
We use Julia as our programming language to implement our examples. This is specially helpful
because Julia is easy to understand and it's \textit{not} made to be ``functional first''. Thus,
we'll have to ``do the work'' in order to construct the concepts of Functional Programming.

These notes are similar to \citet{milewski2018category}, but differ in the sense that we
are both more mathematically formal and (strangely) more pragmatically inclined.
Our examples in Julia aim not only to highlight the category theoretical concept,
but also the strength of Functional Programming.

The notes hop over theory and practice. We always start with the theory and
then showcase the programming examples and how it ties to Functional Programming.
Readers not interested only in the mathematical Category Theory
can skip the programming chapters.


\addcontentsline{toc}{chapter}{Preface}
