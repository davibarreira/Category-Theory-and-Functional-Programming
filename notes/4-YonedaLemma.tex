\chapter{Yoneda Lemma}

One of the most important results in Category Theory is the so called Yoneda Lemma.
In this chapter, we present and prove this lemma. But before going into some
technical details, let's present some intuition behind what's
going on based on the excellent post at Math3ma.

Remember that one of the main points of Category Theory is to describe mathematical
objects via their relations, instead of their definitions. One question that
arises is if we can indeed make the claim that ``an object is
completely determined by its relationships to other objects
''. Of course, the claim is not rigorously posed, so let's dish it out a bit more.

First of, by relationships we are actually saying ``morphisms'',
and we assume that our object $X$ belongs to a category $\mathcal C$.
So, the question is whether $(X, \{\text{definition}\})$ is ``the same''
as $(X, \{\text{relationships of }X\})$. The idea of ``same'' can
be rigorously defined using categorical isomorphism, and the
set of $X$ and it's relationships can actually be written as

\begin{equation}
	(X, \{ f \in Mor_\mathcal C(X,Z) \cup Mor_\mathcal C(Z,X): Z \in Ob_\mathcal C).
	\label{eq:relationdef}
\end{equation}

Let's now give an example of what we are trying to show. Consider the category $\mathbf{Set}$
and two objects $X:=\{1,2\}$ and $Y :=\{a,b\}$. Note that we are introducing these objects
by presenting their definitions, e.g. $X:=\{1,2\}$ is the same as saying
``$\text{X is a set with the elements 1 and 2}$'', which is the same as:
\begin{displaymath}
	(X, \{\text{set with elements 1 and 2}\}).
\end{displaymath}
Our question is if instead of
defining $X$ and $Y$ by describing what they are ``made of'', we could just
define them by describing the many ways they relate to other sets, i.e. by describing their relations.

To study this question, we need a better way to describe the object and it's relationships
instead of trying to work with \eqref{eq:relationdef}. The way to do this is
by noting that we can actually turn this into a functor, which is
a functorial representation of an object.

This representation is not that straight forward, and will require
some more definitions along the way.

\begin{definition}[Representable Functor]
	Let $\mathcal C$ be a locally small category and $A \in Ob_\mathcal C$.
	Define a functor
	\begin{displaymath}
		H^A \equiv Mor_\mathcal C(A,-): \mathcal C \to \mathbf{Set},
	\end{displaymath}

	such that for $B \in Ob_\mathcal C$, we have $Mor_\mathcal C (A,-)(B) \equiv Mor_\mathcal C (A,B)$, and
	for $ f \in Mor_\mathcal C (B,C)$ we have
	\begin{gather*}
		H^A f \equiv Mor_\mathcal C(A,f) : Mor_\mathcal C(A,B) \to Mor_\mathcal C(A,C), \text{ where} \\
		\forall g \in Mor_\mathcal C(A,B) , \ H^A f (g) = f \circ g.
	\end{gather*}

	It's clear that $H^A$ is a functor. If a functor $F:\mathcal C \to \mathbf{Set}$
	is naturally isomorphic to $H^A$ for some $A \in \mathcal C$, then we say that
	$F$ is \textbf{representable}. Note that only functors with $\mathbf{Set}$
	as codomain are possible to be representable.
\end{definition}
Note that the definition requires $\mathcal C$ to be locally small in order to $Mor_\mathcal C (A,B) \in Ob_{\mathbf{Set}}$.

Let's now look to the family of representable functors in a category $\mathcal C$, i.e.
$(H^A)_{A \in Ob_\mathcal C}$. We can define a \textit{contravariant} functor
$H^\bullet$ that takes an object $A \in Ob_\mathcal C$ to $H^A$, and for
a morphism $f \in Mor_\mathcal C (A,B)$, $H^\bullet f \equiv H^f : H^B \to H^A$ is a natural
transformation where
\begin{gather*}
	H^f(D):
	\underbrace{H^B(D)}_{Mor_\mathcal C(B,D)} \to
	\underbrace{H^A(D)}_{Mor_\mathcal C(A,D)}, \\
	\text{such that }\forall p \in Mor_\mathcal C (B,D), \ H^f (D) (p) = p \circ f.
\end{gather*}

Thus, note that the contravariant functor $H^\bullet$ has domain $\mathcal C$
and the codomain is the category of functors from $\mathcal C$ to $\mathbf{Set}$.
Instead of working with the contravariant functor, we can use lemma
\ref{lemma:contravariant}, and instead use the equivalent version where
$H^\bullet$ is a covariant functor, thus concluding that
$H^\bullet: \mathcal C^{op} \to \mathbf{Set}^\mathcal C$.

\begin{lemma}[Yoneda Lemma]
	Let $\mathcal C$ be a category, and $F:\mathcal C \to \mathbf{Set}$ a functor.
	Therefore, for every $A \in \mathcal C$, there exists
	\begin{displaymath}
		\theta_{F,A}: \text{Nat}(Mor_\mathcal C (A, -), F) \to FA,
	\end{displaymath}
	where $\theta_{F,A}$ is a bijetive function from the natural transformations
	between the functors $Mor_\mathcal C (A, -)$ and $F$, and the set of elements of the set $FA$
	\footnote{Note that it's implicit here the fact that the class of natural transformation between
		these two functors is actually a set.}.
	Moreover, the bijections $\theta_F := (\theta_{F,A})_{A \in Ob_\mathcal C}$ constitute a natural transformation
	in the variable $A$; and when $\mathcal C$ is a small category,
	then $(\theta_{F,A})_{F \in Fun(\mathcal C, \mathbf{Set})}$
	consitutes a natural transformation in the variable $F$.
\end{lemma}

Let's prove that the definition above is indeed a category.
For that, we need to define a way to compose natural transformations that
is associative, and we also need to define an identity morphism.

\begin{proposition}
	In the category $\mathcal D^{\mathcal C}$, define the composition of two natural
	transformations $\alpha:F\to G$ and $\beta:G\to H$ as
	\begin{displaymath}
		\forall c \in \mathcal C, \ \alpha_c \fatsemi \beta_c = (\alpha \fatsemi \beta)_c.
	\end{displaymath}
	Then, using this definition, $\alpha \fatsemi \beta$ is indeed a normal transformation
	from $F \to H$, and this composition is associative.
\end{proposition}
\begin{proof}
	First, since
	$\alpha_c \in Mor_\mathcal D (F(c),G(c))$,
	$\beta_c \in Mor_\mathcal D (G(c),H(c))$,
	$\gamma_c \in Mor_\mathcal D (H(c),J(c))$,
	then
	\begin{align*}
		\alpha \fatsemi (\beta \fatsemi \gamma) \iff
		\forall c \in \mathcal C, \alpha_c \fatsemi (\beta \fatsemi \gamma)_c & =
		\alpha_c \fatsemi (\beta_c \fatsemi \gamma_c)                             \\&=
		(\alpha_c \fatsemi \beta_c) \fatsemi \gamma_c                             \\&=
		(\alpha_c \fatsemi \beta_c) \fatsemi \gamma_c                             \\&=
		(\alpha \fatsemi \beta)_c \fatsemi \gamma_c
		\iff (\alpha \fatsemi \beta) \fatsemi \gamma.
	\end{align*}
	We proved the associative part. Now, we need to show that such $\alpha \fatsemi \beta$
	is a natural transformation from $F$ to $H$. Take $f:c\to c' \in Mor_\mathcal C$.
	Consider $(\alpha \fatsemi \beta)_{c}:F(c) \to H(c)$, and
	$(\alpha\fatsemi \beta)_{c'}:F(c') \to H(c')$. Note that
	\begin{displaymath}
		F(f) \fatsemi \alpha_{c'} = \alpha_{c} \fatsemi G(f), \quad
		G(f) \fatsemi \beta_{c'} = \beta_{c} \fatsemi H(f).
	\end{displaymath}

	Therefore, we have
	\begin{align*}
		F(f) \fatsemi (\alpha\fatsemi \beta)_{c'} =
		(F(f) \fatsemi \alpha_{c'}) \fatsemi \beta_{c'} & =
		(\alpha_c \fatsemi G(f)) \fatsemi \beta_{c'}        \\&=
		\alpha_c \fatsemi (G(f) \fatsemi \beta_{c'})        \\&=
		\alpha_c \fatsemi \beta_c \fatsemi H(f)             \\&=
		(\alpha \fatsemi \beta)_c \fatsemi H(f).
	\end{align*}
\end{proof}

