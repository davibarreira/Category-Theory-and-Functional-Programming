\chapter{Limits, Colimits}

Limits and colimits are universal constructions that generalize some of the
categorical concepts we already talked about, such as products,
equalizers, pullbacks and pushouts.
Limits have a terminal sort of universal properties.
Colimits have an initial sort of universal properties.

They are of interest, because whenever a category is complete with
respect to limits or colimits, these end up corresponding to some
classical concept.

\section{Cones and Cocones}

Before talking about limits, we need to introduce the notion of a cone and it's dual, the cocone.

\begin{definition}[Cone]
	Let $\mathcal C$ be a \textit{small} category. The (left) cone $\mathcal C^{\lhd}$ is a category constructed
	by adding an object $LC$, called the cone point, into the set of objects of $\mathcal C$,
	and making it an initial object by guaranteeing that there is only one morphism $l_A$ from $LC$ to each
	$A \in Ob_\mathcal C$. In other words, $\mathcal C^{\lhd}$ is a category where:
	\begin{align*}
		Ob(\mathcal C^{\lhd})        & := \{LC\} \sqcup Ob(\mathcal C), \\
		Mor_{\mathcal C^{\lhd}}(A,B) & :=
		\begin{cases}
			Mor_\mathcal C (A,B)  \quad & \text{if } A,B \in Ob(\mathcal C),       \\
			\{l_A\}  \quad              & \text{if } A = LC, B \in Ob(\mathcal C), \\
			\{id_{LC}\}  \quad          & \text{if } A = LC, B = LC,               \\
			\varnothing \quad           & \text{if } A \in Ob(\mathcal C), B = LC.
		\end{cases}
	\end{align*}
\end{definition}

As we pointed out, the cocone is just the dual.

\begin{definition}[Cocone (Right Rone)]
	Let $\mathcal C$ be a \textit{small} category. The cocone $\mathcal C^{\rhd}$ is a category constructed
	by adding an object $RC$, called the cone point, into the set of objects of $\mathcal C$,
	and making it an initial object by guaranteeing that there is only one morphism $r_A$ from $A \in Ob_\mathcal C$ to each
	$RC$. In other words, $\mathcal C^{\rhd}$ is a category where:
	\begin{align*}
		Ob(\mathcal C^{\rhd})        & := \{RC\} \sqcup Ob(\mathcal C), \\
		Mor_{\mathcal C^{\rhd}}(A,B) & :=
		\begin{cases}
			Mor_\mathcal C (A,B)  \quad & \text{if } A,B \in Ob(\mathcal C),       \\
			\{r_B\} \quad               & \text{if } A = RC, B \in Ob(\mathcal C), \\
			\{id_{RC}\}  \quad          & \text{if } A = RC, B = RC,               \\
			\varnothing  \quad          & \text{if } A = RC, B \in Ob(\mathcal C), \\
		\end{cases}
	\end{align*}
\end{definition}

\textbf{Caution!} When constructing, for example, the cone category, there can
be only a single morphism leaving $LC$. One might be tempted to construct this
category by drawing an arrow from the initial object $LC$ to every object
in the category $\mathcal C$, but this might lead to more than one morphism
from $LC$ to an object.

\begin{example}[Cone in Discrete Categories]
	Let $\mathbf{\underline{3}}$ be the discrete category with three objects.
\end{example}

\begin{example}[Cone in $\mathbf{Gr}$]
	Let $\mathbf{\underline{3}}$ be the discrete category with three objects.
\end{example}


\section{Category of Cones}

A category of cones is a way to define categories of certain diagrams (functors) inside another category.
The definition by itself is very obscure, so we instead present how to construct the category of
cones for \textit{spans} (remember definition \ref{def:Span} and Figure \ref{fig:Span}).
This example makes clear the motivation behind the slice category.

\begin{example}[Category of Spans in $\mathcal C$]
	Consider a diagram $D:\underline{2}\to \mathcal C$. Note that
	the cone $\underline{2}^{\lhd}$ is a span as shown in the Figure below.

	(Figure of span from $\underline{2}^{\lhd}$)

	Now, consider the category $\mathcal C^{\underline{2}^{\lhd}}$, i.e.
	the category of functors $F:\underline{2}^{\lhd} \to \mathcal C$,
	where the morphisms are natural transformations.
	Each functor $F \in Ob_{\mathcal C^{\underline{2}^{\lhd}}}$
	is a span inside category $\mathcal C$, as shown in the Figure below.

	(Figure here)

	The next step is to formally define the category of spans between two given objects,
	e.g. we want to formally define a category of spans in $A, B \in Ob_\mathcal C$,
	where each object is a functor $F:\underline{2}^{\lhd}\to \mathcal C$,
	such that $F(1) = A$ and $F(2) = B$.

	The caveat here is that we have to do this in a categorical way, without actually applying restrictions
	to the object specifications. To do this, we define a functor
	$X:\underline{2} \to \mathcal C$, where $X(1) = A$ and $X(2)= B$.

	Next, we define an inclusion functor $i: \underline{2} \to \underline{2}^{\lhd}$ which sends
	everything in $\underline{2}$ to the same point, but in the bigger category $\underline{2}^{\lhd}$
	(similar to how we can define $i: \mathbb N \to \mathbb R$ where $i(1) = 1.0$).

	We can now construct the following diagram:

	Thus, if $F\circ i = X$, then the diagram above commutes and $F(1) = A, F(2) = B$, which means
	that $F$ is a span of $A$ and $B$. This is the categorical way to restrict our permissible functors
	to spans of $A,B$. Therefore, we define the category of spans between $A,B$ as
	$\mathcal C_{/ X}$, where
	\begin{displaymath}
		Ob_{\mathcal C_{/ X}}:=\{
		F: \underline{2}^\lhd \to \mathcal C \ | \ F \circ i = X
		\}.
	\end{displaymath}
	\begin{displaymath}
		Mor_{\mathcal C_{/X}}(F,G):=\{
		\alpha: F \Rightarrow G \ | \ \alpha \diamond i = id_{X}
		\}.
	\end{displaymath}
	Note that $\alpha$ is a natural transformation, and $\alpha \diamond i$ is the \textit{prewhiskering}
	operation \ref{def:prewhiskering}. The composition between morphisms is defined as in the category of functors.

	We already explained the motivation for why the definition of $Ob_{\mathcal C_{/X}}$
	is restricted to $F\circ i = X$,
	but not why $\alpha \diamond i = id_{X}$ in the morphisms. This comes from the fact
	that for $\alpha:F \Rightarrow G$, then
	$\alpha \diamond i: F \circ i \Rightarrow G \circ i$. Since
	$F$ and $G$ are objects in $\mathcal C_{/X}$, we have
	$F \circ i = X$ and $G \circ i = X$, implying that
	$\alpha \diamond i: X \Rightarrow X$, i.e. $\alpha \diamond i = 1_{X}$.

	This proves that indeed such restriction is what we want for morphisms. The intuition
	is just that we want a natural transformation between spans $A, B$ to exists
	only if $\alpha$ sends $A$ to $A$ and $B$ to $B$, as shown in the figure below.

	Surprisingly, the product $(A\times B, \pi_1, \pi_2)$ is the terminal object
	of the category of spans between $A$ and $B$, and, as we'll see, this means
	that the product is the limit of $X:\underline{2} \to \mathcal C$.
\end{example}

So, in the example above, we've constructed the category of spans between $A,B$,
which consisted the category of functors $S:\underline{2}^\lhd \to C$, restricted
to $S \circ i = X$, where $X: \underline{2}\to \mathcal C$ is a diagram
on the discrete category of two elements with $X(1) = A, X(2) = B$.
We can therefore generalize this construction
by considering an arbitrary category $I$ instead of $\underline{2}$,
and any functor ($I$-shaped diagram) $X:I\to \mathcal C$.

\begin{definition}[Slice Category]
	Let $\mathcal C$ and $\mathcal I$ be two categories, and $i:I \to I^{\lhd}$ the
	inclusion functor. For a given functor $X:I \to \mathcal C$, the slice category
	of $\mathcal C$ over $X$ is denoted $\mathcal C_{/X}$, where:

	\begin{displaymath}
		Ob_{\mathcal C_{/ X}}:=\{
		F: I^\lhd \to \mathcal C \ | \ F \circ i = X
		\}.
	\end{displaymath}
	\begin{displaymath}
		Mor_{\mathcal C_{/X}}(F,G):=\{
		\alpha: F \Rightarrow G \ | \ \alpha \diamond i = id_{X}
		\}.
	\end{displaymath}
\end{definition}

By taking the dual definition, we can define the \textit{coslice}.

\begin{definition}[Coslice Category]
	Let $\mathcal C$ and $\mathcal I$ be two categories, and $i:I \to I^{\rhd}$ the
	inclusion functor. For a given functor $X:I \to \mathcal C$, the coslice category
	of $\mathcal C$ over $X$ is denoted $\mathcal C_{X/}$, where:

	\begin{displaymath}
		Ob_{\mathcal C_{X/}}:=\{
		F: I^\rhd \to \mathcal C \ | \ F \circ i = X
		\}.
	\end{displaymath}
	\begin{displaymath}
		Mor_{\mathcal C_{X/}}(F,G):=\{
		\alpha: F \Rightarrow G \ | \ \alpha \diamond i = id_{X}
		\}.
	\end{displaymath}
\end{definition}

\section{Defining Limits and Colimits}

With the definition of slices and coslices, the definition of a limit and
colimit is immediate.
\begin{definition}[Limit]
	Let $\mathcal C$ and $\mathcal I$ be two categories, and $X:I \to \mathcal C$
	a functor. The limit of $X$ is a terminal object in the slice category
	$\mathcal C_{/X}$, which is denoted by $\lim X$ or $\lim_I X$.
\end{definition}

Remember that a category can have several terminal objects, but they will all
be isomorphic, with a unique isomorphism between them. Thus, in a sense, a limit
will always be unique up to an isomorphism.

Again, we can easily define a dual definition.

\begin{definition}[Colimit]
	Let $\mathcal C$ and $\mathcal I$ be two categories, and $X:I \to \mathcal C$
	a functor. The limit of $X$ is a initial object in the coslice category
	$\mathcal C_{X/}$, which is denoted by $\text{colim} X$ or $\text{colim}_I X$.
\end{definition}
