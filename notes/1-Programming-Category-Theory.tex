\chapter{Categories and Functional Programming}

Now that we've defined what a category is according to Category Theory,
one might wonder how this relates to programming. In this chapter,
we introduce Functional Programming,
and we show how FP and CT are related.

\section{Practical Introduction to Functional Programming}

According to \citet{widman2022learning},
there are three ``main'' programming paradigms: imperative programming, Object-Oriented Programming (OOP), and Functional Programming (FP).
Imperative programming is \textit{plain} programming, in the sense that you define variables, and functions, and mutate it's values.
OOP models programs via objects. An object is from a class, and it has a state and methods. Thus, the value mutation is usually
encapsulated by in the state of an object.
Finally, FP tries to eradicate value mutation. Once a value is assigned to a variable, this should not change.
This might seem strange, but the idea is that value mutation is usually a source of bugs and complications, thus,
by limiting it, we can try to make our code more robust. This property is usually called ``immutability''.
Besides immutability, \citet{widman2022learning} lists four other core tenants of Functional Programming,
which are referential transparency,
pure functions, higher order functions and lazy evaluations.

In a sense, Functional Programming is the most unusual of the three approaches, as value mutation
is a very natural way of thinking in code. Yet, by avoiding value mutation, functions
in FP have a closer resemblance to what we call functions in mathematics (more specifically in Set Theory).

The definition that we gave for Functional Programming was pragmatic. One important
aspect missing from such definition was the necessity of working with data types.
Data types are at the core of how FP relates to Category Theory.
A more formal definition of Functional Programming is given further in the chapter.
In order to do this, we will need to introduce Type Theory.

\subsection{Is Julia an FP language?}

As stated in the preface, we are going to use
Julia as the language to code our examples.
Julia is \textit{not} as FP oriented as languages such as Haskell and Scala.
This helps people that are not used to FP. Also,
the language enforces types and is \textit{not} Object-Oriented, which eases
the process of porting FP concepts. In summary, Julia stands close enough to FP
without the often unusual syntax of highly FP oriented languages.

\subsection{Julia's Type System}

These notes do not intend to introduce the Julia programming language, as there are
already plenty of resources on the subject. Yet, it is worthy to say some brief words on how
Julia deals with types.

In Julia, types can be concrete or abstract. Only concrete type can be instantiated, while abstract
types are ways of grouping these types. For example, \pc{Int} and \pc{Float64} are concrete types, while
\pc{Number} is an abstract type that contains both, i.e. \pc{Int} and \pc{Float64} are subtypes of \pc{Number}.
We can check whether a type is a subtype of another type via the \pc{<:} operator, as shown below.
\bigskip
\begin{lstlisting}[language=JuliaLocal, style=julia, texcl=true]
julia> Int <: Number, Float64 <: Number
(true, true)
\end{lstlisting}

Types can be enforced when calling arguments in a functions, e.g.
\pc{f(x::Int, y::String)}. Yet, the output type of a function is not enforced.
This type enforcing allows for a feature called multiple dispatch. In Julia,
a function can be defined several times by varying only the arguments, both
quantity and types. Thus, we can have something like:

\begin{lstlisting}[language=JuliaLocal, style=julia, texcl=true]
f(x::Int) = x^2
f(x::Int,y) = (x^2, y)
f(x::String) = "a string!"
f(x::Int, y::Int) = x + y

julia> f(2)
4
julia> f(2,"ok")
(4,"ok")
julia> f("test")
a string!
julia> f(1,1)
2
\end{lstlisting}

Note that the same function \pc{f} was defined several times, one for each
dispatching argument. Each of these instances is called a method of the function \pc{f}.
When we don't define the type of the argument,
Julia uses the type \pc{Any}, which means any type.
If a function has a method with a more specific type, the compiler tries to
call the more specialized function. In our example, the method \pc{f(x::Int, y::Int)}
was more specialized than \pc{f(x::Int, y)}. This idea of a more specialized method
is possible due to the fact that types have a hierarchy as we have shown in the
example with types \pc{Int}, \pc{Float64} and \pc{Numbers}.

Besides the default types that Julia provides, we can create new types.
There are several ways of doing this, such as structs. By creating new types,
we can define functions that take variables with these new types as arguments
and dispatch on them.

\begin{lstlisting}[language=JuliaLocal, style=julia, texcl=true]
struct Point2D
    x::Real
    y::Real
end

julia> p = Points2D(1,1)
Point2D(1, 1)
julia> p.x
1
julia> getproperty(p, :x)
1
\end{lstlisting}

Structs are by default immutable, which is good for FP. Yet, we can define mutable structs by simply
adding the word ``mutable'' before the ``struct''.

Here is a more interesting example of how to use structs, type hierarchy and multiple-dispatch:

\begin{lstlisting}[language=JuliaLocal, style=julia, texcl=true]
abstract type Shape end

struct Point3D
    x::Real
    y::Real
    z::Real
end

struct Square <: Shape
    center::Point3D
    length::Real
    Square(center, length) = length ≥ 0 ? new(center, length) : error("length should be greater or equal than 0.")
end

struct Circle <: Shape
    center::Point3D
    radius::Real
    Circle(center, radius) = radius ≥ 0 ? new(center, radius) : error("radius should be greater or equal than 0.")
end

area(s::Shape) = 0.0
area(s::Circle) = π * s.radius^2
area(s::Square) = length^2

julia> s = Square(10,Point3D(0,0,0));
julia> area(s)
100
\end{lstlisting}
In our example, we set the default area of a shape to zero by defining our \pc{area(s::Shape)} equal
to \pc{0.0}. For the
shapes that we know how to compute the area, we write the formulas (e.g. square and circle). Note that
we wrote \pc{Square <: Shape} to indicate that both of our structs are subtypes of \pc{Shape}.
Note how Julia is \textit{not} Object-Oriented. The type \pc{Square} and the type \pc{Circle}
do not have methods themselves. We instead define functions that dispatch on the desired type.
This illustrates how Julia is somewhat FP oriented, but not all the way, since we don't have
output type enforcing like in, for example, Haskell.


\subsection{A First Example of Immutability}

As we've said, FP has as a core principle the idea of immutable variables.
If we consider a For-Loop, in Python, a simple\lstinline[style=julia]{}\lstinline[style=julia]{}for i in range(0,10):print(i)\lstinline[style=julia]{}\lstinline[style=julia]{}
will create a variable \lstinline[style=julia]{i} that mutates by taking first value 0 then 1, and so on.
In Julia, the default behaviour of a For-Loop is different. Julia enforces local scope,
and creates local variables that are then destroyed. This is more like an FP style.
\bigskip
\begin{lstlisting}[language=JuliaLocal, style=julia, texcl=false]
for i in 1:4
    m = 1
end
\end{lstlisting}
Instead of using a For-Loop, a functional way of doing this would be by defining a function and recursively calling it.
\bigskip
\begin{lstlisting}[language=JuliaLocal, style=julia, texcl=true]
function myloop(i::Int)
    if i > 5
        return
    end
    println(i)
    return myloop(i+1)
end

julia> myloop(0)
0
1
2
3
4
5
\end{lstlisting}

\subsection{Side Effects, Pure Functions and Referential Transparency}

Consider a function \pc{f}. A function produces a side effect if it affects anything outside of it's scope. For example:
\bigskip
\begin{lstlisting}[language=JuliaLocal, style=julia, texcl=true, escapeinside={(@}{@)}]
x = []
function f!(x)
    push!(x,1)
end

julia> f!(x);
julia> x
(@Any[1]@)
\end{lstlisting}

In Julia, we usually write an exclamation to the end of the name of a function
to indicate that this function has side effects.
Note that this is \textit{just} a notation.
The exclamation point does not modify the function.
A function without side effects is called \textit{pure}.

Another important aspect is ``referencial transparency''.
This means that for a given input \pc{x} a function should always return the same output \pc{y}.
This seems logical, but there are clear examples that break this. For example:
\bigskip
\begin{lstlisting}[language=JuliaLocal, style=julia, texcl=true, escapeinside={(@}{@)}]
julia> rand(1)
(@1-element Vector{Float64}:@)
(@ 0.024980886821626025@)
\end{lstlisting}
Note that, in Mathematics,
what we define as a function usually follows these two principles,
i.e. mathematical functions are pure and referentially transparent.

\subsection{Higher Order Functions}

A high order function is a function that receives functions as input or that returns a function as output.
In FP, we want functions to be ``first-class citizens'', meaning that we can pass them to variables and to other
functions just like a regular value.
Here is one example:

\bigskip
\begin{lstlisting}[language=JuliaLocal, style=julia, texcl=true, escapeinside={(@}{@)}]
square() = nums -> map(x->x^2, nums)

julia> square()(10)
100
\end{lstlisting}

The function \pc{square} when called actually returns a function.
Another similar consists in
\bigskip
\begin{lstlisting}[language=JuliaLocal, style=julia, texcl=true, escapeinside={(@}{@)}]
square_plus_one = function(x)
    x^2 + 1
end

julia> square_plus_one(1)
2

square_plus_two = x->x^2+2

julia> square_plus_two(2)
6
\end{lstlisting}

\subsection{Lazy Evaluation}

Another relevant concept is the one about lazy evaluation.
In Julia we don't natively have lazy evaluation, on the contrary,
our code is \textit{eagerly} evaluated.
This means that once we call a function, it evaluates all the parameters.
Of course, we can alter our code to try to make it lazy. Consider the following example:

\bigskip
\begin{lstlisting}[language=JuliaLocal, style=julia, texcl=true]
imap = Iterators.map # version of `map` that returns an iterable
take = Iterators.take # returns the `n` first values of an iterable.
squarelazy(nums) = imap(x->x+1,nums)

julia> x = 1:10;
julia> squarelazy(x);
julia> x
1:10
\end{lstlisting}

Note that our function did not evaluate x. Lastly, we use the \pc{collect}
function to actually evaluate our iterator.
\bigskip
\begin{lstlisting}[language=JuliaLocal, style=julia, texcl=true]
julia> collect((squarelazy ∘ take)(x,2))
2-element Vector{Int64}:
 2
 3
\end{lstlisting}

\section{Categories in Programming}

The emphasis of FP in controlling side-effects,
having referentially transparent functions and enforcing types
makes programming functions similar to functions in set-theory.
We know that in Category Theory we have a category $\mathbf{Set}$,
where sets are objects and functions are morphisms. Using $\mathbf{Set}$
as an analogy, we can define a category $\mathbf{Prog}$, where
types are objects and pure referentially transparent functions are morphisms.
Thus, the analogy between $\mathbf{Prog}$ and $\mathbf{Set}$ is clear.
Each type corresponds to a set and each programming function corresponds to a set-theoretic function.

Note that this correspondence is not perfect.
Functions in programming, even if pure and referentially transparent,
might still be non-terminating; and deciding whether a function is non-terminating is undecidable (halting problem).
Hence, programming functions cannot be equated to functions in set-theory. As pointed out by
\citet{milewski2018category}, a possible solution is adding a value \textit{bottom}
to each type as to represent the possibility of non-termination. Yet, this causes other
theoretical problems.

From a practical point of view, we follow \citet{milewski2018category} and we ignore
non-terminating functions. Thus, our category $\mathbf{Prog}$ can be thought as a
subcategory of $\mathbf{Set}$\footnote{We claim that $\mathbf{Prog}$ is only subcategory
of $\mathbf{Set}$ because not necessarily all sets have a corresponding type. Consider
for example that type \pc{Int} is not actually $\mathbb Z$, but only the finite set
of numbers going from $-2^{63}$ to $2^{63}-1$.} with each type corresponding to set, and each programming function
corresponding to a set-theoretic function.

As we've pointed out, Julia does not enforce functions to be pure or referentially transparent.
Julia also does check if two composing functions match input and output types. Thus, this is left
for us programmers to enforce. If the functions we define follow these theoretical assumptions
(i.e. our functions are pure, referentially transparent, terminate, and
only compose with functions that match domain and codomain), then we can assume that
we are working in the category $\mathbf{Prog}$.

Once we stablish that we are in the category $\mathbf{Prog}$, we have successfully
connected the world of programming and the world of mathematics.
This means that we can now use mathematical tools to prove properties
about our programs. We can also import concepts from Category
Theory and use it for programming purposes. In fact, Functional Programming does just that.
Many software patterns of Functional Programming are based on Category Theory.
Functors, Monoids and Monads are some of the many examples.


\subsection{Exploring $\mathbf{Prog}(J)$ with Julia}

Note that our previous explanation of $\mathbf{Prog}$ did not define
what types are encompassed on $\mathbf{Prog}$. The types
present actually depends on the programming language we are considering.
It's possible that a programming language has a type that represents real numbers,
while other does not. Also, there are programming languages that have the possibility
of defining new types, while other might not. Hence, it's more accurate
to define a category as $\mathbf{Prog}(L)$, where $L$ is the prog

First, let's give a definition for our category.
\begin{definition}[$\mathbf{Prog}$]
	Category $\mathbf{Prog}$ is a subcategory of $\mathbf{Set}$ where programming
	types are the objects and correspond to a set. The morphisms are pure
	referentially transparent and terminating functions, which correspond to
	a function in $\mathbf{Set}$.
\end{definition}

Since $\mathbf{Prog}$ is a subcategory of $\mathbf{Set}$, let's present
some of the properties of $\mathbf{Set}$ and show their correspondence in $\mathbf{Prog}$.
First, the initial object of $\mathbf{Set}$ is the empty set.
In Julia, the initial object is the type \pc{::Union\{\}}.
This type is also known as \pc{Base.Bottom}, and it's a subtype of every type in Julia, including
itself. Thus, it behaves just like $\varnothing$, in the sense that $\varnothing \subset A$
for any set $A$ including $\varnothing$.

\bigskip
\begin{lstlisting}[language=JuliaLocal, style=julia, texcl=true]
julia> Union{} <: Int, Union{} <: Nothing, Union{} <: Union{}
(true, true, true)
\end{lstlisting}

Note that a function $f:\varnothing \to A$ cannot ever be called, since we cannot
provide an element of $\varnothing$. In the same way, a function
\pc{f(x::Union\{\})} cannot be called, because there is no instance of type \pc{Union\{\}}.

The terminal object of $\mathbf{Set}$ are the singleton sets. Note that they all are initial objects,
but they are all isomorphic.
In Julia, the \pc{Nothing} type is a singleton type, which is the value \pc{nothing}.
So is the type \pc{Tuple\{\}}, which only has \pc{()} as element.
We can also define our own terminal types using a struct:

\bigskip
\begin{lstlisting}[language=JuliaLocal, style=julia, texcl=true]
struct Terminal end

julia> Base.issingletontype(Terminal)
True
\end{lstlisting}
The only element of our type \pc{Terminal} is \pc{Terminal()}.
