\chapter{Abstract Algebra}

% This is based partially on \citet{garling2011clifford}
Mostly based on \citet{aluffi2021algebra}.

\section{Initial Definitions for Groups}

\begin{definition}[Groups]
	Consider the triple $(G, \cdot, e)$, where $G$ is a set,
	$\cdot : G \times G \to G$ is the product mapping
	and $e \in G$ is the identity element.
	This triple is a group if:
	\begin{enumerate}
		\item (Associativity): $a \cdot (b \cdot c) = (a \cdot b) \cdot c$ for every $a,b,c \in G$;
		\item (Identity): $a \cdot e = e \cdot a  = a$ for every $a \in G$;
		\item (Inverse): For every $a \in G$ there exists $a^{-1} \in G$ such that
		      $a\cdot a^{-1} = a^{-1}\cdot a = a$;
	\end{enumerate}

	When there is no ambiguity, we call the set $G$ a group omitting
	the product and neutral element.
\end{definition}

Whenever it's not ambiguous, we omit the product operator,
thus, $g\cdot h \equiv g h$.

\begin{definition}[Abelian Group]
	A group $(G, \cdot, e)$ is \textit{Abelian} if besides the group properties
	(i.e. associativity, identity and inverse)
	it's also commutative, i.e. $a \cdot b = b \cdot a$ for every $a,b \in G$.
\end{definition}

\begin{example}
	Note that $(\mathbb R, +, 0)$ is an Abelian Group.
	In this case, $a^{-1}$ is usally denoted as $-a$.
	The triple $(\mathbb R \setminus \{0\}, \cdot, 1)$ is also an Abelian Group.

	An example of non-Abelian group would be the set of invertible
	matrices from $\mathbb R^n$ to $\mathbb R^n$,
	with $\cdot$ as matrix composition, e.g.
	$A \cdot B = A B$. Since every matrix considered is invertible
	and we have the identity matrix as our identity element,
	then we indeed have a non-Abelian group,
	since the matrix product is not commutative.
\end{example}

\begin{proposition}[Group Cancellation]
	Let $(G, \cdot)$ be a group. Therefore:
	\begin{displaymath}
		fa = ha \implies f = h, \quad \quad
		af = ah \implies f = h
	\end{displaymath}
\end{proposition}
\begin{proof}
	If $fa = ha$, then $fa a^{-1} = h a ^a{-1} \implies f = h$.
\end{proof}

\begin{definition}[Subgroup Generated]
	Let $(G, \cdot, e)$ be a group. We say that $S\subset G$ is a
	subgroup of $G$ if $(S, \cdot, e)$ is a group.
	For $A \subset G$, $\text{Gr}(A)$ is called the subgroup generated by $A$,
	and it's the smallest subgroup of $G$ containing $A$, i.e.
	$\cap_{\alpha \in \Gamma} S_\alpha$ where $\{S_\alpha\}_{\alpha \in \Gamma}$ are
	all the sets that are subgroups of $G$.
	It's easy to prove that such set is indeed a subgroup.

	For a singleton $\{g\}$, we define
	$\text{Gr}(g) := \{g^n \ : \ n \in \mathbb Z\}$, where
	$g^0 = e$, and $g^n$ is the product of $n$ copies of $g$,
	while $g^{-n}$ is the product of $n$ copies of $-g$.
\end{definition}

\begin{definition}[Cyclic Group]
	If a group $G$ is equal to $\text{Gr}(g)$ for some $g \in G$,
	then we say that $G$ is cyclic.
\end{definition}

\begin{definition}[Order of Groups]
	The order of a finite group $G$ is the number of elements of $G$.
	An element $g \in G$ has \textit{finite order} if $g^n = e$
	for $n \in \mathbb N$. The order of $g$ is then
	the smallest $n$ such that $g^n = e$.
\end{definition}

\begin{definition}[Homomorphism and Isomorphism]
	Let $(G, \cdot_G, e_G)$ and $(H, \cdot_H, e_H)$ be two groups. A function
	$\theta:G \to H$ is a homomorphism between $G$ and $H$ if
	$\theta(g_1 \cdot_G g_2) = \theta(g_1) \cdot_H \theta(g_2)$
	for every $g_1, g_2 \in G$.

	If $\theta$ is bijective, then we say that $\theta$
	is an isomorphism.
\end{definition}

\begin{definition}[Normal / Self-conjugate]
	Let $K$ be a subgroup of $G$. We say that $K$ is \textit{normal},
	or \textit{self-conjugate}, if $g k g^{-1} \in K$ for every $g \in G$.
\end{definition}

\section{Groups and Category Theory}

Remember that in Category Theory we have a notion of isomorphism
that generalizes set isomorphism (i.e. bijective function between sets).

\begin{definition}[Automorphism]
	Let $A$ be an object of a category $\mathcal C$. An automorphism
	is an isomorphism from $A$ to itself. The set\footnote{Remember
		that a $Hom(A,A)$ is guaranteed to be a set if the category is locally small.}
	of automorphism of $A$ is denoted by $\text{Aut}_\mathcal C(A)$.
\end{definition}

\begin{definition}[Grupoid and Groups]
	A groupoid is a category where every morphism is an isomorphism.
	Hence, a group is a groupoid category with a single object $G$.
	We denote $\textbf{Grp}$ as the catogory of groups. In similar
	fashion, we can define $\textbf{Ab}$ as the category of abelian
	groups, where the only difference is that the objects are abelian
	groups.
\end{definition}

Note that this definition is equivalent to our definition of
a group in algebraic terms. Why? Because every morphism
is equivalent to an element of $G$, and the morphism composition
does the part of the product operator. Also, note that
every category has an identity morphism, thus,
$id_G \equiv e$ our neutral element. Since every morphism
is an isomorphism, this means that for every $g \in Hom(G,G)$,
there is a $g^{-1} \in Hom(G,G)$ such that $g \circ g^{-1} = id_G = e$.

In pure categorical terms. Let $\mathcal C$ be a locally small category
and $G \in \mathcal C$, i.e. an object of $\mathcal C$.

\section{Rings and Modules}

Let's begin by remembering the concept of a monoid. A monoid $(M, \cdot, e)$
is a set $M$, together with the binary operator $\cdot: M \times M \to M$ and the identity
element $e$. Besides, $\cdot$ is associative.

\begin{definition}[Ring]
    A ring $(R, \cdot, +)$ is an abelian group $(R,+)$ together with a monoid $(R, \cdot)$,
    with the property of distributivity, i.e. $a \cdot (b + c) = a\cdot b + a \cdot c$ for every
    $a,b,c \in R$.

    One usually denotes the identity of $(R,+)$ by $0_R$ and the identity of $(R, \cdot)$
    by $1_R$. The reason is clear, since these are the corresponding identities
    for the usual sum and multiplication of numbers.\footnote{Note that sometimes this is called
    a unital ring. Another definition of ring would be to consider $(R, \cdot,+)$ as a semigroup
instead of a monoid, which would imply the inexistence of an identity element.
A ring that is not unital is sometimes called a Rng.}
\end{definition}

Based on this definition, one can prove that:
\begin{proposition}
    Let $(R, \cdot, +)$ be a ring. Therefore:
    \begin{displaymath}
        0 \cdot r = 0 = r \cdot 0.
    \end{displaymath}
\end{proposition}

Note that in this definition, the $+$ operator has much more stated properties, e.g.
there are inverse elements, there is commutativity. The $\cdot$ has more freedom.
For example, we are not requiring for an inverse to exist, and neither commutativity.
Which leads to the following definition.

\begin{definition}[Commutative Ring]
	A ring $(R, \cdot, +)$ is commutative if $a \cdot b = b \cdot a$ for every $a,b \in R$.
\end{definition}

Now, we want to slowly increment the properties of these algebraic concepts in order
to construct our usual suspects, e.g. $\mathbb N$ $\mathbb Q$, $\mathbb R$, $\mathbb Q$ and $\mathbb C$.

\begin{definition}[Zero-Divisor]
	Let $(R,\cdot, +)$ be a ring. We say that $a \in R$ is a left-zero-divisor
	if there exists $b \neq 0 \in R$ such that $a b = 0$. Analogously, we define
	a right-zero-divisor.
\end{definition}

Note that $0 \in R$ is a zero-divisor of every ring $R$ \textbf{with the exception}
of the \textit{zero-ring} case. The zero-ring is the ring where $R$ is a singleton
set. Hence, since there is only one element, there is no element such that
$a b = 0$ for $b\neq 0$, since no such $b$ exists.

\begin{example}[$\mathbb Z \setminus n \mathbb Z$]
	Let $n$ be a positive integer.
\end{example}

\begin{definition}[$R$-Module]
	An abelian group $(M, \oplus)$ is called a module
	over a ring $(R,+,\cdot)$ if there is a map (often called
	scalar multiplication) where:
	\begin{displaymath}
		*:R \times M \to M,
	\end{displaymath}
	such that for all $r,r' \in R$ and $m,m' \in M$ we have
	\begin{enumerate}[(i)]
		\item $0_R * m = 0_M$;
		\item $1_R * m = m$;
		\item $(r+r') * m = r*m \oplus r'*m$;
		\item $r * (m \oplus m') = r*m \oplus r*m'$;
		\item $(r \cdot r')*m = r * (r' * m)$.
	\end{enumerate}

	We also call this an $R$-Module $M$.
\end{definition}

\begin{definition}[$R$-Algebra]
	An $R$-Algebra $M$ is an $R$-Module $M$ together with a
	bilinear map $M\times M \to M$.
\end{definition}

Note that a vector space $V$ over $\mathbb R$
with an inner product is an example of $R$-algebra.

% We already know that they are groups with respect to the sum operator. Yet, the
% \textit{groupness} is just part of their definition. They also have multiplication
% with the distributivity property which makes them to be rings.

% Yet again, the \textit{rigness} is not a complete description, because we also know
% that, for example, the $0$ element 
