\newpage

\chapter{Programming with Categories and Functional Programming}
In programming languages, we can define a category by considering types to be objectsand functions to be morphisms.


\section{1.1 Types in Julia}
In Julia, types can be concrete or abstract. Only concrete type can be instantiated, while abstracttypes are ways of grouping these types. For example, \lstinline[style=julia]{Int} and \lstinline[style=julia]{Float64} are concrete types, while\lstinline[style=julia]{Number} is an abstract type that contains both, i.e. \lstinline[style=julia]{Int} and \lstinline[style=julia]{Float64} are subtypes of \lstinline[style=julia]{Number}.
\bigskip
\begin{lstlisting}[language=JuliaLocal, style=julia, texcl=true]
using Pkg
Pkg.activate(".")
\end{lstlisting}

\begingroup

\fontsize{10pt}{12pt}\selectfont

\begin{verbatim}
  Activating project at `~/MEGA/EMAP/PhDThesis/notes/FunctionalProgrammingCategoryTheory/julia`

\end{verbatim}

\endgroup

\bigskip
\begin{lstlisting}[language=JuliaLocal, style=julia, texcl=true]
Int <: Number, Float64 <: Number
\end{lstlisting}

\begingroup

\fontsize{10pt}{12pt}\selectfont

\begin{verbatim}

(true, true)
\end{verbatim}

\endgroup

\bigskip
\begin{lstlisting}[language=JuliaLocal, style=julia, texcl=true]
isconcretetype(Int)
\end{lstlisting}

\begingroup

\fontsize{10pt}{12pt}\selectfont

\begin{verbatim}

true
\end{verbatim}

\endgroup

\bigskip
\begin{lstlisting}[language=JuliaLocal, style=julia, texcl=true]
isabstracttype(Real)
\end{lstlisting}

\begingroup

\fontsize{10pt}{12pt}\selectfont

\begin{verbatim}

true
\end{verbatim}

\endgroup
It's easy to create an abstract type in Julia.Types can also be composite or primitive. A primitive type is a concrete type where it's definition is given in bits.Primitive types are the ones you would imagine, e.g. \lstinline[style=julia]{Float16}, \lstinline[style=julia]{Float32}, \lstinline[style=julia]{Float64}, \lstinline[style=julia]{Bool}, \lstinline[style=julia]{Int8} and so on.

In Julia, one can define his own primitive type, but this is not usually optimal, since it's not native to the LLVM (the compiler structure used in Julia).

Composite types are types composed of other types. In Julia, these are the so called \lstinline[style=julia]{structs}.
\bigskip
\begin{lstlisting}[language=JuliaLocal, style=julia, texcl=true]
MyType <: MyTypes
\end{lstlisting}

\begingroup

\fontsize{10pt}{12pt}\selectfont

\begin{verbatim}

true
\end{verbatim}

\endgroup
Another way of constructing types is with operations such as \lstinline[style=julia]{Union}.
\bigskip
\begin{lstlisting}[language=JuliaLocal, style=julia, texcl=true]
IntOrNothing = Union{Int, Nothing}

1 isa IntOrNothing
\end{lstlisting}

\begingroup

\fontsize{10pt}{12pt}\selectfont

\begin{verbatim}

true
\end{verbatim}

\endgroup
Another possible contruction is with \lstinline[style=julia]{Tuple}.
\bigskip
\begin{lstlisting}[language=JuliaLocal, style=julia, texcl=true]
Tuple2Int = Tuple{Int,Int}

(1,2) isa Tuple2Int
\end{lstlisting}

\begingroup

\fontsize{10pt}{12pt}\selectfont

\begin{verbatim}

true
\end{verbatim}

\endgroup

\bigskip
\begin{lstlisting}[language=JuliaLocal, style=julia, texcl=true]
MyType2 = Tuple{MyType,MyType}

(MyType(1,2,3),MyType(3,3,"o")) isa MyType2
\end{lstlisting}

\begingroup

\fontsize{10pt}{12pt}\selectfont

\begin{verbatim}

true
\end{verbatim}

\endgroup
Another interesting way of creating types with tuples is by using \lstinline[style=julia]{Vararg}. In this case, wecan do:
\bigskip
\begin{lstlisting}[language=JuliaLocal, style=julia, texcl=true]
MyTypeTuple = Tuple{MyType, Vararg{MyType}}

(MyType(1,2,3),MyType(1,2,3),MyType(3,3,"o")) isa MyTypeTuple
\end{lstlisting}

\begingroup

\fontsize{10pt}{12pt}\selectfont

\begin{verbatim}

true
\end{verbatim}

\endgroup
Another very important aspect of types are the so called parametric types.Note that we just created a plethora of new types with the caveat thatthey all have the fields \lstinline[style=julia]{x} and \lstinline[style=julia]{y}, and both fields have the same type.
\bigskip
\begin{lstlisting}[language=JuliaLocal, style=julia, texcl=true]
NewType(10,10)
\end{lstlisting}

\begingroup

\fontsize{10pt}{12pt}\selectfont

\begin{verbatim}

NewType{Int64}(10, 10)
\end{verbatim}

\endgroup
You can also specify a supertype for the parameter, for example:
\bigskip
\begin{lstlisting}[language=JuliaLocal, style=julia, texcl=true]
struct NewType2{T<:Real}
    x::T
    y::T
end

function f(x::NewType2{T}) where T <: Int
    x.x+x.y
end
\end{lstlisting}

\begingroup

\fontsize{10pt}{12pt}\selectfont

\begin{verbatim}

f (generic function with 2 methods)
\end{verbatim}

\endgroup
An important thing to note here is that in parametric types is that subtyping behaves a bit different thanfor non-composite types. See:
\bigskip
\begin{lstlisting}[language=JuliaLocal, style=julia, texcl=true]
@show NewType(10,10) isa NewType{Int}
@show NewType(10,10) isa NewType{Real};
\end{lstlisting}

\begingroup

\fontsize{10pt}{12pt}\selectfont

\begin{verbatim}
NewType(10, 10) isa NewType{Int} = true
NewType(10, 10) isa NewType{Real} = false

\end{verbatim}

\endgroup
Although \lstinline[style=julia]{Int <: Real}, it's not true that \lstinline[style=julia]{NewType{Int} <: NewType{Real}}. Yet, we can saythat \lstinline[style=julia]{NewType{Int} <: NewType}.
\bigskip
\begin{lstlisting}[language=JuliaLocal, style=julia, texcl=true]
NewType{Int} <: NewType
\end{lstlisting}

\begingroup

\fontsize{10pt}{12pt}\selectfont

\begin{verbatim}

true
\end{verbatim}

\endgroup
Here the use of tuples can be helpful. Tuples allow the use of parameters, and the subtyping works for the elements also.
\bigskip
\begin{lstlisting}[language=JuliaLocal, style=julia, texcl=true]
Tuple{Int, Int} <: Tuple{Real, Real}
\end{lstlisting}

\begingroup

\fontsize{10pt}{12pt}\selectfont

\begin{verbatim}

true
\end{verbatim}

\endgroup
Lastly, note the following:
\bigskip
\begin{lstlisting}[language=JuliaLocal, style=julia, texcl=true]
@show typeof(NewType{Int})
@show typeof(NewType)
@show DataType <: Type;
\end{lstlisting}

\begingroup

\fontsize{10pt}{12pt}\selectfont

\begin{verbatim}
typeof(NewType{Int}) = DataType
typeof(NewType) = UnionAll
DataType <: Type = true

\end{verbatim}

\endgroup
In Julia, even types have types!Moreover, what is the difference between a DataType and UnionAll?You can think that a DataType can infer how much memory will be needed for the type, whilea UnionType can't. Why?

Note that for \lstinline[style=julia]{NewType{Int}} we know that two ints will have to be stored in memory, hencewe know the amount of bits necessary. Yet, for \lstinline[style=julia]{NewType}, we cannot predict how much bits will be used,since the parameter type could be pretty much anything.

But what about \lstinline[style=julia]{Union{Int,Float64}} and \lstinline[style=julia]{Tuple{Int,Float64}?}
\bigskip
\begin{lstlisting}[language=JuliaLocal, style=julia, texcl=true]
@show typeof(Union{Int,Float64})
@show typeof(Union{Int,Real})
@show typeof(Tuple{Int,Float64});
\end{lstlisting}

\begingroup

\fontsize{10pt}{12pt}\selectfont

\begin{verbatim}
typeof(Union{Int, Float64}) = Union
typeof(Union{Int, Real}) = DataType
typeof(Tuple{Int, Float64}) = DataType

\end{verbatim}

\endgroup
Note that in the first example, we have a type \lstinline[style=julia]{Union}, which makes sense, since we are either going to allocate thesize of an Int or a Float. But what about the second union? In the second case, Julia is just turning\lstinline[style=julia]{Union{Int,Real}} in the type \lstinline[style=julia]{Real}, sort of just saying "allocate the larger memmory.

Lastly, the Tuple is also a DataType, since we know how much each element will store.
\section{1.2 Value Types}


Types in Julia are very useful due to multiple-dispatch. It would be very useful if one could also dispatch on values,for example, a function \lstinline[style=julia]{f(x::Bool)} with two methods, one for \lstinline[style=julia]{x == True} and another for \lstinline[style=julia]{x == False}.

This can be done using \lstinline[style=julia]{Val}. This is already implemented in Julia, but here is how one could implement it.
\bigskip
\begin{lstlisting}[language=JuliaLocal, style=julia, texcl=true]
struct MyVal{x}
end
MyVal(x) = MyVal{x}()

@show MyVal(1);
\end{lstlisting}

\begingroup

\fontsize{10pt}{12pt}\selectfont

\begin{verbatim}
MyVal(1) = MyVal{1}()

\end{verbatim}

\endgroup

\bigskip
\begin{lstlisting}[language=JuliaLocal, style=julia, texcl=true]
f(::Val{true}) = 1
f(::Val{false}) = 0

f(Val(true)) ,f(Val(false))
\end{lstlisting}

\begingroup

\fontsize{10pt}{12pt}\selectfont

\begin{verbatim}

(1, 0)
\end{verbatim}

\endgroup
Note that this is easily abused, and must be used carefully.The proper way to use it is such that the compiler knows what is going to be passed.
\section{1.3 Initial and Terminal Objects}
In Category Theory, the initial object is one where there is a unique morphismleaving it to any other object. In the case o $\mathbf{Set}$, this is the empty set, sinceevery morphism from the empty set is the "same" uncallable function.

In Julia, the initial object is the type \lstinline[style=julia]{::Union{}}.Note that this type is known as \lstinline[style=julia]{Base.Bottom}, and it's actually a subtype of every type in Julia.
\bigskip
\begin{lstlisting}[language=JuliaLocal, style=julia, texcl=true]
Union{} <: Int, Union{} <: Float64, Union{} <: Nothing
\end{lstlisting}

\begingroup

\fontsize{10pt}{12pt}\selectfont

\begin{verbatim}

(true, true, true)
\end{verbatim}

\endgroup
Since \lstinline[style=julia]{Union{}} is a subtype of every other type, it behaves in the same way as the empty set,which is a subset of every other set.\textit{Ex falso sequitur quodlibet}.The exact opposite of \lstinline[style=julia]{Union{}} is the \lstinline[style=julia]{Any} type, i.e. any value is of type \lstinline[style=julia]{Any}.Now, in the other end of the spectrum, we have a terminal object, which is an object thatfor every object in the category, there is only one morphism arriving to it.

In $\mathbf{Set}$, this terminal object is the singleton set, i.e., the set with a single object.Now, you may wonder, "but we have an infinite number of singletons, which one is the terminal?". The answer isthat they all are "the same" up to an isomorphism.Hence, the terminal object is the unique up to an isomorphism.

But what about in Julia? The terminal object would be a type that has only a single possible "value", alsocalled a singleton type. Hence,\lstinline[style=julia]{Int} or \lstinline[style=julia]{Float64} or \lstinline[style=julia]{Char} are not possibilities, since they have many elements.We could construct such a type. See:
\bigskip
\begin{lstlisting}[language=JuliaLocal, style=julia, texcl=true]
struct Terminal end

Terminal()
\end{lstlisting}

\begingroup

\fontsize{10pt}{12pt}\selectfont

\begin{verbatim}

Terminal()
\end{verbatim}

\endgroup

\bigskip
\begin{lstlisting}[language=JuliaLocal, style=julia, texcl=true]
Base.issingletontype(Terminal)
\end{lstlisting}

\begingroup

\fontsize{10pt}{12pt}\selectfont

\begin{verbatim}

true
\end{verbatim}

\endgroup
In Julia, the \lstinline[style=julia]{Nothing} type is already a type with a singleton, which is the value \lstinline[style=julia]{nothing}.So is the \lstinline[style=julia]{Tuple{}} type, which only has \lstinline[style=julia]{()} as element.
\bigskip
\begin{lstlisting}[language=JuliaLocal, style=julia, texcl=true]
Base.issingletontype(Nothing),Base.issingletontype(Tuple{})
\end{lstlisting}

\begingroup

\fontsize{10pt}{12pt}\selectfont

\begin{verbatim}

(true, true)
\end{verbatim}

\endgroup

\section{1.4 Product and Coproduct of Types}


As we already talked about, types are objects in programming languages such as Julia, and we cancreate product objects and coproduct objects, i.e. product types and coproduct types.

So what are they? We actually already talked about them. The \lstinline[style=julia]{Tuple{Type1, Type2}} is the productof \lstinline[style=julia]{Type1} and \lstinline[style=julia]{Type2}, while \lstinline[style=julia]{Union{Type1,Type2}} is the coproudct.
\bigskip
\begin{lstlisting}[language=JuliaLocal, style=julia, texcl=true]
Tuple{Int,Int}
\end{lstlisting}

\begingroup

\fontsize{10pt}{12pt}\selectfont

\begin{verbatim}

Tuple{Int64, Int64}
\end{verbatim}

\endgroup

\section{1.5 Type Classes and Traits in Julia}


As we said, Julia can defined types in a hierarchical nature, but it does not provide a"native" way of grouping distincting types in classes.Remember, we could say that some types belong to the same class if they share some \textbf{traits}.

The benefit of having a class of types is that once we know that a certain type pertainsto a certain class, we can then use what we know about such class.For example, we could think of a class \lstinline[style=julia]{Eq} which would consist of all types in Juliawhere the \lstinline[style=julia]{==} operator is defined.
\bigskip
\begin{lstlisting}[language=JuliaLocal, style=julia, texcl=true]
struct Point2
    x::Real
    y::Real
end

Base.isequal(p::Point2, q::Point2)::Bool = p.x == q.x && p.y == q.y ? true : false

p = Point2(1,1)
q = Point2(1,1)
q_ = Point2(2,1)

p == q, p == q_
\end{lstlisting}

\begingroup

\fontsize{10pt}{12pt}\selectfont

\begin{verbatim}

(true, false)
\end{verbatim}

\endgroup
In the implementation above, we've defined a new type \lstinline[style=julia]{Point2}, and such type would now belong to the class \lstinline[style=julia]{Eq}.The question is, how can we define such class in Julia?

What we need is a way to group types that go beyond the simple type hierarchy. To do this,we use a design pattern known (in Julia) as Holy Trait (note that "Holy" is the surname of Tim Holy, the creator of this idea).In the code above, we've implemented the class \lstinline[style=julia]{Eq}.First, we used \lstinline[style=julia]{Eq(::Type)= IsNotEq()}, which means that, by default, a type does NOT belong to the \lstinline[style=julia]{Eq} class.Next, we used \lstinline[style=julia]{Eq(::Type{<:Point2})= IsEq()} and others to specify that such types indeed were from class \lstinline[style=julia]{Eq}.At first this might not look very useful, yet, it can be, specially if you think of multiple-dispatch.

Consider the following example.Now, we want to implement a function such as \lstinline[style=julia]{area} that iscomputed by multiplying \lstinline[style=julia]{l*h}. Note that all shapes have this property, but \lstinline[style=julia]{Circle}.Yet, all of them share the \lstinline[style=julia]{Geomtric} type, hence, we cannot just dispatch on it.

How to solve this? We could perhpas create an \lstinline[style=julia]{if x is Circle}, but this is not very good,cause we could add a new geometric shape like \lstinline[style=julia]{Ellipse} which would also not have the \lstinline[style=julia]{l} and \lstinline[style=julia]{h}...
\bigskip
\begin{lstlisting}[language=JuliaLocal, style=julia, texcl=true]
HasHeightLengthTrait(::Type) = NotHasHeightLength()
HasHeightLengthTrait(::Type{<:Rect})   = HasHeightLength()
HasHeightLengthTrait(::Type{<:Trapz})  = HasHeightLength()
HasHeightLengthTrait(::Type{<:Square}) = HasHeightLength()
\end{lstlisting}

\begingroup

\fontsize{10pt}{12pt}\selectfont

\begin{verbatim}

HasHeightLengthTrait
\end{verbatim}

\endgroup

\bigskip
\begin{lstlisting}[language=JuliaLocal, style=julia, texcl=true]
hasheightlength(x::T) where T<: Geometric  = hasheightlength(HasHeightLengthTrait(T), x)
hasheightlength(::HasHeightLength, x) = true
hasheightlength(::NotHasHeightLength, x) = false
\end{lstlisting}

\begingroup

\fontsize{10pt}{12pt}\selectfont

\begin{verbatim}

hasheightlength (generic function with 3 methods)
\end{verbatim}

\endgroup

\bigskip
\begin{lstlisting}[language=JuliaLocal, style=julia, texcl=true]
area(x::T) where T<: Geometric  = area(HasHeightLengthTrait(T), x)
area(::HasHeightLength, x) = x.l * x.h
area(::NotHasHeightLength, x) = error("Area function not implemented.")

s = Square(1)
area(s)
\end{lstlisting}

\begingroup

\fontsize{10pt}{12pt}\selectfont

\begin{verbatim}

1
\end{verbatim}

\endgroup

\section{1.6 Enumerations}


The @enum macro create an enumeartion struct. This is nothing more than a type suit wherethe possible values are enumerated. Look the example below.
\bigskip
\begin{lstlisting}[language=JuliaLocal, style=julia, texcl=true]
@enum suit clubs diamonds hearts spades
suit
\end{lstlisting}

\begingroup

\fontsize{10pt}{12pt}\selectfont

\begin{verbatim}

Enum suit:
clubs = 0
diamonds = 1
hearts = 2
spades = 3
\end{verbatim}

\endgroup

\bigskip
\begin{lstlisting}[language=JuliaLocal, style=julia, texcl=true]
instances(suit)
suit(1), Int(diamonds)
\end{lstlisting}

\begingroup

\fontsize{10pt}{12pt}\selectfont

\begin{verbatim}

(diamonds, 1)
\end{verbatim}

\endgroup
Note that, for example, \lstinline[style=julia]{diamonds} is of type \lstinline[style=julia]{suit}, and it's not a type itself, but an instance of \lstinline[style=julia]{suit}. Similarto how  \lstinline[style=julia]{1} is an instance of \lstinline[style=julia]{Int}.
\section{1.7 What it means to be a value of type T?}


As pointed out, in programming we have a category $\textbf{Types}$ where types are objects and functionsare morphisms.If \lstinline[style=julia]{Int} is an object, then what is a value of \lstinline[style=julia]{Int} represented in Category Theory?

One might be tempted to say "well, it's just an element of the object". The problem is that, in Category Theory,we are not "allowed" to peek inside objects. We can only talk about morphisms, objects, and things of higher abstraction (e.g. functors).

To solve this, note that we can think of a type \lstinline[style=julia]{T} as a set, and we can show thatthere is an isomorphism between the values of \lstinline[style=julia]{T} and the set of morphisms (functions) fromthe terminal type \lstinline[style=julia]{Terminal} to type \lstinline[style=julia]{T}. Why? Remember, a terminal type \lstinline[style=julia]{Terminal} isa type containing only a singleton value. In the example of \lstinline[style=julia]{Terminal}, such singleton is \lstinline[style=julia]{Terminal()}.

Hence, for each \lstinline[style=julia]{a::A}, we can define a function \lstinline[style=julia]{fa(x::Terminal) = a}. Since the only valueof \lstinline[style=julia]{Terminal} is \lstinline[style=julia]{Terminal()}, this means that each function from \lstinline[style=julia]{Terminal} to \lstinline[style=julia]{A} is exactlya function that takes \lstinline[style=julia]{Terminal()} to a value of \lstinline[style=julia]{A}. Therefore, the set of morphisms from\lstinline[style=julia]{Terminal} to \lstinline[style=julia]{A} is isomorphic to the set of values of \lstinline[style=julia]{A}.

Now, consider the following code:
\bigskip
\begin{lstlisting}[language=JuliaLocal, style=julia, texcl=true]
struct T
    x::Int
    y::Int
end
a = (1,2)
t = T(a...)
\end{lstlisting}

\begingroup

\fontsize{10pt}{12pt}\selectfont

\begin{verbatim}

T(1, 2)
\end{verbatim}

\endgroup
When we created such struct, Julia implicitly created a "value" constructor, i.e.a function \lstinline[style=julia]{T(a::Tuple{Int,Int})} that returns values of type \lstinline[style=julia]{T}.Since a value \lstinline[style=julia]{a::Tuple{Int,Int}} is just a function \lstinline[style=julia]{a(x::Terminal)::Tuple{Int,Int}},then \lstinline[style=julia]{T(a)} is actually a function composition \lstinline[style=julia]{T ∘ a} in the category $\mathbf{Types}$.

Thus, we are able to talk about \lstinline[style=julia]{T(1,2)} as a morphism (function), without requiring to talkabout elements of sets.

But what about the fields \lstinline[style=julia]{x} and \lstinline[style=julia]{y}? For a value of type \lstinline[style=julia]{t = T(a)}, \lstinline[style=julia]{t.x} can be understoodas a function \lstinline[style=julia]{x(t::T)::Int}, and the same for \lstinline[style=julia]{y}. Note that\lstinline[style=julia]{t} is isomorphic to \lstinline[style=julia]{T ∘ a} , thus\lstinline[style=julia]{x(t)} is ismorphic to \lstinline[style=julia]{x ∘ t = x ∘ T ∘ a}.

Hence, again we've defined \lstinline[style=julia]{t.x} in terms of a morphism.

Lastly, we can prove that our type \lstinline[style=julia]{T} is isomorphic to \lstinline[style=julia]{Tuple{Int,Int}}.Two objects are isormophic if there is a morphism (function) which is left and right invertible,meaning, for  \lstinline[style=julia]{f(x::Tuple{Int,Int})::T}, there is \lstinline[style=julia]{fl(y::T)::Tuple{Int,Int}} and\lstinline[style=julia]{fr(y::T)::Tuple{Int,Int}} such that:


\begin{displaymath}
	f_l\circ f = id_{\text{Tuple\{Int,Int\}}}
\end{displaymath}



\begin{displaymath}
	f\circ f_r = id_{\text{T}}
\end{displaymath}


In our example, this function is just \lstinline[style=julia]{f(x,y) = T(x,y)}. The inverse of \lstinline[style=julia]{f}is \lstinline[style=julia]{fl(T) = (T.x, T.y)}.