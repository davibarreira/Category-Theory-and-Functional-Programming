\newpage

\chapter{**Functional Programming - Introduction**}


This is based on the book "Learning Functionl Programming" (Jack Widman).


\section{1. What is Functional Programming?}


There are three "main" programming paradigms: imperative programmming, Object-Oriented Programming (OOP), and Functional Programming (FP).Imperativie programming is \textit{plain} programming, in the sense that you define variables, and functions, and mutate it's values.OOP models programs via objects. An object is from a class, and it has a state and methods. Thus, the value mutation is usuallyencapsulated by in the state of an object.Finally, FP tries to erradicate (minimize) value mutation. Once a value is assigned to a varible, this shoudl not change.This might seem strange, but the idea is that value mutation is usually a source of bugs and complications, thus,by limiting it, we can try to make our code more robust.



In a sense, Functional Programming is the most unusual of the three approaches, as value mutationis a very natural way of thinking in code.


\section{2. Is Julia an FP language?}


This question comes up a lot in Julia's forums. "Is Julia a Functional Programming Language?". The answerto this is "not really". For starters, Julia allows value mutation. FP is a paradigm, and there areprogramming languages that are more prone to FP than others. Thus, we can apply Functional Programming ideasto Julia (and even to Python).
\bigskip
\begin{lstlisting}[language=JuliaLocal, style=julia, texcl=true]
using Pkg
Pkg.activate(".")
\end{lstlisting}

\begingroup

\fontsize{10pt}{12pt}\selectfont

\begin{verbatim}
  Activating project at `~/MEGA/EMAP/Julia_Tutorials/FunctionalProgramming`

\end{verbatim}

\endgroup

\section{3. A First Example of Immutability}


As we've said, FP has as a core principle the idea of immutable variables.If we consider a For-Loop, in Python, a simple\lstinline[style=julia]{}\lstinline[style=julia]{}for i in range(0,10):print(i)\lstinline[style=julia]{}\lstinline[style=julia]{}

will create a variable \lstinline[style=julia]{i} that mutates by taking first value 0 then 1, and so on.

In Julia, the default behaviour of a For-Loop is different. Julia enforces localscope, and creates local variables that are then destroyed. This is more likea FP style.
\bigskip
\begin{lstlisting}[language=JuliaLocal, style=julia, texcl=true]
for i in 1:4
    m = 1
end
try
    i
catch
    println("`i` is note defined")
end

try
    m
catch
    println("`m` is note defined")
end
\end{lstlisting}

\begingroup

\fontsize{10pt}{12pt}\selectfont

\begin{verbatim}
`i` is note defined
`m` is note defined

\end{verbatim}

\endgroup
Instead of using a For-Loop, a functional way of doing this wouldbe by defining a function and recursively calling it.
\bigskip
\begin{lstlisting}[language=JuliaLocal, style=julia, texcl=true]
function myloop(i::Int)
    if i > 5
        return
    end
    println(i)
    return myloop(i+1)
end
myloop(0)
\end{lstlisting}

\begingroup

\fontsize{10pt}{12pt}\selectfont

\begin{verbatim}
0
1
2
3
4
5

\end{verbatim}

\endgroup

\section{4. Side Effects, Pure Functions and Referencial Transparency}


Consider a function \lstinline[style=julia]{f}. A function produces a side effect if it affects anything outsideof it's score. For example:
\bigskip
\begin{lstlisting}[language=JuliaLocal, style=julia, texcl=true]
x = []
function f!(x)
    push!(x,1)
end
f!(x)
@show x;
\end{lstlisting}

\begingroup

\fontsize{10pt}{12pt}\selectfont

\begin{verbatim}
x = Any[1]

\end{verbatim}

\endgroup
In Julia, we usually write an exclamation to the end of the name of a functionto indicate that this function has side effects. Note that this is \textit{just} a notation.The exclamation point does not modify the function.A function without side effects is called \textit{pure}.

Another important aspect is "referencial transparency". This means thatfor a given input \lstinline[style=julia]{x} a function should always return the same output \lstinline[style=julia]{y}.This seems logical, but there are clear examples that break this.For example:
\bigskip
\begin{lstlisting}[language=JuliaLocal, style=julia, texcl=true]
rand(1)
\end{lstlisting}

\begingroup

\fontsize{10pt}{12pt}\selectfont

\begin{verbatim}

1-element Vector{Float64}:
 0.024980886821626025
\end{verbatim}

\endgroup
Note that, in Mathematics, what we define as a function usually follows these two principals, i.e.mathematical functions are pure and referencially transparent.

\textbf{Def.(Function in Set Theory)}A function $f$ from a set $X$ to $Y$ is an assignment of an element of $Y$ to each element of $X$,denoted by $f: X \to Y$.The set $X$ is called domain, and $Y$ the codomain. The set $f(X):=\{f(x) :  x \in X\}$ is theimage of $f$.
\section{5. Higher Order Functions}


A high order function is a function that receives functions as input or that returns a function as output.In FP, we want functions to be "first-class citizens", meaninig that we canpass them to variables and to other functions just like a regular value.

Here is one example:
\bigskip
\begin{lstlisting}[language=JuliaLocal, style=julia, texcl=true]
square() = nums -> map(x->x^2, nums)
square()(10)
\end{lstlisting}

\begingroup

\fontsize{10pt}{12pt}\selectfont

\begin{verbatim}

100
\end{verbatim}

\endgroup
The function \lstinline[style=julia]{square} when called actually returns a function.Another similar consists in
\bigskip
\begin{lstlisting}[language=JuliaLocal, style=julia, texcl=true]
square_plus_one = function(x)
    x^2 + 1
end
@show square_plus_one(1)

square_plus_two = x->x^2+2

@show square_plus_two(2);
\end{lstlisting}

\begingroup

\fontsize{10pt}{12pt}\selectfont

\begin{verbatim}
square_plus_one(1) = 2
square_plus_two(2) = 6

\end{verbatim}

\endgroup

\section{6. Lazy Evaluation}


Another relevant concept is the one about lazy evaluation.In Julia we don't natively have lazy evaluation, on the contrary,our code is \textit{eagerly} evaluated. This means that once we call a function,it evaluates all the parameters.

Of course, we can alter our code to try to make it lazy.Consider the following example:
\bigskip
\begin{lstlisting}[language=JuliaLocal, style=julia, texcl=true]
imap = Iterators.map # This is a version of `map` that returns an iterable
take = Iterators.take # This function takes an iterable and returns the `n` first values.
\end{lstlisting}

\begingroup

\fontsize{10pt}{12pt}\selectfont

\begin{verbatim}

take (generic function with 2 methods)
\end{verbatim}

\endgroup

\bigskip
\begin{lstlisting}[language=JuliaLocal, style=julia, texcl=true]
squarelazy(nums) = imap(x->x+1,nums)
x = 1:10

squarelazy(x);
x
\end{lstlisting}

\begingroup

\fontsize{10pt}{12pt}\selectfont

\begin{verbatim}

1:10
\end{verbatim}

\endgroup

\bigskip
\begin{lstlisting}[language=JuliaLocal, style=julia, texcl=true]
typeof(1:10)
\end{lstlisting}

\begingroup

\fontsize{10pt}{12pt}\selectfont

\begin{verbatim}

UnitRange{Int64}
\end{verbatim}

\endgroup
Note that our function did not evaluate x.
\bigskip
\begin{lstlisting}[language=JuliaLocal, style=julia, texcl=true]
collect((squarelazy ∘ take)(x,2))
\end{lstlisting}

\begingroup

\fontsize{10pt}{12pt}\selectfont

\begin{verbatim}

2-element Vector{Int64}:
 2
 3
\end{verbatim}

\endgroup
